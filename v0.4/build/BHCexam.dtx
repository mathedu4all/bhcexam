% \iffalse meta-comment
%
% Copyright (C) 2014 by Charles Bao <charley792@gmail.com>
%
% This file is part of the BHCexam package project.
% ---------------------------------------------------
%
% It may be distributed under the conditions of the LaTeX Project Public
% License, either version 1.2 of this license or (at your option) any
% later version. The latest version of this license is in
%    http://www.latex-project.org/lppl.txt
% and version 1.2 or later is part of all distributions of LaTeX
% version 1999/12/01 or later.
%
%<*!(cfg|fd)>
% \fi
%
%% \CharacterTable
%%  {Upper-case    \A\B\C\D\E\F\G\H\I\J\K\L\M\N\O\P\Q\R\S\T\U\V\W\X\Y\Z
%%   Lower-case    \a\b\c\d\e\f\g\h\i\j\k\l\m\n\o\p\q\r\s\t\u\v\w\x\y\z
%%   Digits        \0\1\2\3\4\5\6\7\8\9
%%   Exclamation   \!     Double quote  \"     Hash (number) \#
%%   Dollar        \$     Percent       \%     Ampersand     \&
%%   Acute accent  \'     Left paren    \(     Right paren   \)
%%   Asterisk      \*     Plus          \+     Comma         \,
%%   Minus         \-     Point         \.     Solidus       \/
%%   Colon         \:     Semicolon     \;     Less than     \<
%%   Equals        \=     Greater than  \>     Question mark \?
%%   Commercial at \@     Left bracket  \[     Backslash     \\
%%   Right bracket \]     Circumflex    \^     Underscore    \_
%%   Grave accent  \`     Left brace    \{     Vertical bar  \|
%%   Right brace   \}     Tilde         \~}
%%
%
% \CheckSum{0}
%
% \iffalse meta-comment
%</!(cfg|fd)>
%
%<*driver>
\ProvidesFile{BHCexam.dtx}
%</driver>
%
%<cls>\NeedsTeXFormat{LaTeX2e}[1995/12/01]
%<cls>\ProvidesClass{BHCexam}
%<cfg>\ProvidesFile{BHCexam.cfg}
  [2015/10/10 v0.4 BHCexam
%<cls>   document class]
%<cfg>   configuration file]
%
%<*driver>
   bundle source file]
%</driver>
%
%<*driver>
\documentclass[a4paper]{ltxdoc}
\usepackage{ctex}
\usepackage{hyperref}
\usepackage{amsmath,amssymb}
 \topmargin 0.5 true cm
 \oddsidemargin 1 true cm
 \evensidemargin 1 true cm
 \textheight 21 true cm
 \textwidth 14 true cm
\EnableCrossrefs
 %\DisableCrossrefs % Say \DisableCrossrefs if index is ready
\CodelineIndex
\RecordChanges      % Gather update information
 %\OnlyDescription  % comment out for implementation details
 %\OldMakeindex     % use if your MakeIndex is pre-v2.9
\hypersetup{colorlinks,linkcolor=blue,citecolor=blue}
\begin{document}  
 \DocInput{BHCexam.dtx}
\end{document}
%</driver>
%
% \fi
%
%
% \changes{v0.0}{2011/07/22}{Initial version}
% \changes{v0.1}{2011/07/23}{version 0.1}
% \changes{v0.2}{2011/07/27}{version 0.2}
% \changes{v0.3}{2014/02/18}{version 0.3}
% \changes{v0.3}{2015/10/10}{version 0.4}
%
%
% \DoNotIndex{\begin,\end,\begingroup,\endgroup}
% \DoNotIndex{\ifx,\ifdim,\ifnum,\ifcase,\else,\or,\fi}
% \DoNotIndex{\let,\def,\xdef,\newcommand,\renewcommand}
% \DoNotIndex{\expandafter,\csname,\endcsname,\relax,\protect}
% \DoNotIndex{\Huge,\huge,\LARGE,\Large,\large,\normalsize}
% \DoNotIndex{\small,\footnotesize,\scriptsize,\tiny}
% \DoNotIndex{\normalfont,\bfseries,\slshape,\interlinepenalty}
% \DoNotIndex{\hfil,\par,\vskip,\vspace,\quad}
% \DoNotIndex{\centering,\raggedright}
% \DoNotIndex{\c@secnumdepth,\@startsection,\@setfontsize}
% \DoNotIndex{\ ,\@plus,\@minus,\p@,\z@,\@m,\@M,\@ne,\m@ne}
% \DoNotIndex{\@@par}
%
%
% \GetFileInfo{BHCexam.dtx}
%
%
% \MakeShortVerb{\|}
% \setcounter{StandardModuleDepth}{1}
%
%
% \newcommand{\ctex}{\texttt{ctex}}
% \newcommand{\BHCexam}{\texttt{BHCexam}}
% \newcommand{\exam}{\texttt{exam}}
% \newcommand{\colin}{\texttt{colinexam}}
%
%
% \setlength{\parskip}{0.75ex plus .2ex minus .5ex}
% \renewcommand{\baselinestretch}{1.2}
%
% \makeatletter
% \def\parg#1{\mbox{$\langle${\it #1\/}$\rangle$}}
% \def\@smarg#1{{\tt\string{}\parg{#1}{\tt\string}}}
% \def\@marg#1{{\tt\string{}{\rm #1}{\tt\string}}}
% \def\marg{\@ifstar\@smarg\@marg}
% \def\@soarg#1{{\tt[}\parg{#1}{\tt]}}
% \def\@oarg#1{{\tt[}{\rm #1}{\tt]}}
% \def\oarg{\@ifstar\@soarg\@oarg}
% \makeatother
%
%
% \title{\bf \BHCexam~宏包说明\thanks
%   {这是鲍宏昌发布的第一个~\LaTeX~宏包。本文件版本号为~\fileversion{},最后修改日期~\filedate{}。}}
% \author{\it 鲍宏昌\thanks{charley792@gmail.com}}
% \date{\small 打印日期:~\today}
% \maketitle
%
% \begin{abstract}
% \BHCexam~宏包提供了一个中学试卷排版的~\LaTeX{}~文档类。
%
% \BHCexam~主要文件包括~\texttt{BHCexam.cls}~文档类和配置文件
% ~\texttt{BHCexam.cfg}。
%
% \BHCexam~宏包由鲍宏昌制作并负责维护。
% \end{abstract}
%
% \tableofcontents
% \newpage
%
% \section{简介}
%
% 本宏包以~\exam~为底层文档类,部分源代码来自于盖鹤麟开发的
% ~\colin。不知道什么原因盖鹤麟自2004年就一直没有发布更新,
% ~\colin~仍然使用CCT实现中文支持,而缺乏对~XeTeX~的支持。2011年7月,
% 鲍宏昌在~\colin~的基础上改用~\ctex~实现中文支持,采用UTF8编码同时
% 支持~XeLaTeX~和~pdfTeX~进行编译,并使用~\texttt{doc}~和
% ~\texttt{docstrip}~工具编写了这个文档,增加了一些新的功能,
% 并把新的宏包命名为~\BHCexam。
%
% 本宏包延续了~\colin~和~\exam~的特点,能让一个刚刚接触
% ~\LaTeX~的初学者,也能轻松用它来排版试卷。希望~\BHCexam~能提高中学
% 教师的工作效率,并把注意力放在试卷的内容上。
% 
% \BHCexam~由两个主要文件构成:文档类~\texttt{BHCexam.cls}~和配置文件
% ~\texttt{BHCexam.cfg}~。后者定义了一些常用的参数。
%
% {\kaishu
% 这两个文件可以通过用~XeLaTeX~编译~\texttt{BHCexam.ins}~文件来得到,
% 而这份说明文档可以通过用~XeLaTeX~编译~\texttt{BHCexam.dtx}~文件来得到。
% 编译说明文档需要~\ctex{}~宏包,为了生成正确的索引和版本记录,
% 需要使用如下命令}
% \begin{verbatim}
% makeindex -s gind.ist -o BHCexam.ind BHCexam.idx
% makeindex -s gglo.ist -o BHCexam.gls BHCexam.glo
% \end{verbatim}
%
% \section{一个简单的例子}
%
% 用~\BHCexam~要排版一张基本的试卷其实很简单。如果你准备对试卷的排版进行
% 更细致的设置,那么请参考~\exam~的文档。
%
% \subsection{\texttt{documentclass} 命令}
% \label{sec:BasicDocumentclass}
%
% 要使用~\BHCexam~文档类,你的\verb"\documentclass" 命令应该是
% \begin{verbatim}
%  \documentclass{BHCexam}
% \end{verbatim}
% 如果,你想使用小四字体作为缺省字体大小,那么添加选项\verb"cs4size"
% \begin{verbatim}
% \documentclass[cs4size]{BHCexam}
% \end{verbatim}
% 更多的选项,请参考\ref{sec:Options}。
%
% \subsection{打印标题和考试须知}
% \DescribeMacro{\maketitle}
% \DescribeMacro{\notice}
% 在试卷上打印标题和考试须知
% \begin{verbatim}
%   \maketitle
%   \notice 
% \end{verbatim}
% 关于标题和考试须知中变量的设置,请参考\ref{sec:Variable}。
%
% \subsection{题目}
% \label{sec:Example}
% \DescribeMacro{\question}
% \DescribeMacro{\stk}
% \DescribeMacro{\onech}
% \DescribeMacro{\part}
% 在\verb"questions"环境中用\verb"\question"输入题目。
% 在\verb"parts"环境中用\verb"\part"输入大题的小问。
% 用\verb"\stk"、\verb"\mtk"
% 和\verb"\ltk"输入填空题的答案。
% 用\verb"\onech"、\verb"\twoch"
% 和\verb"\fourch"输入选择题的选项。
% 在\verb"questions"环境中用\verb"\tiankong"、\verb"\xuanze"
% 和\verb"\jianda"分别显示填空题、选择题、简答题的提示语。
% \begin{verbatim}
% \begin{questions}
%   \tiankong
%   \question 这是第1道填空题\stk{答案,不显示答案时显示段横线}
%   \question 这是第2道填空题\mtk{答案,不显示答案时显示段横线}
%   \question 这是第2道填空题\ltk{答案,不显示答案时显示段横线}
%   \xuanze
%   \question 问题3是一道选择题,四个选项显示在一行
%   \onech{选项1}{选项2}{选项3}{选项4}
%   \question 问题3是一道选择题,四个选项显示在两行
%   \twoch{选项1}{选项2}{选项3}{选项4}
%   \question 问题3是一道选择题,四个选项显示在四行
%   \fourch{选项1}{选项2}{选项3}{选项4}
%   \jianda
%   \question 问题4是一道简答题
%   \begin{parts}
%   \part 第1小问
%   \part 第2小问
%   \end{parts}
% \end{questions}
% \end{verbatim}
% 关于填空题、选择题、简答题的提示语中几个变量的设置,请参考\ref{sec:Variable}。
% 关于题目的更多内容,请参考\ref{sec:Environment}。
%
% \section{使用帮助}
%
% \subsection{选项}
% \label{sec:Options}
%
% \changes{v0.2}{2011/07/27}{增加UTF8选项以支持pdflatex}
% \changes{v0.4}{2015/10/10}{取消UTF8选项放弃支持pdflatex}

% 宏包的选项用于改变一些缺省的设置。虽然缺省的设置基本能过满足一般用户的
% 使用需要,但用户也可以根据自己的情况,使用这些选项。
%
% \begin{description}
% \item[cs4size]     使用小四字号为缺省字体大小。
% \item[c5size]      使用五号字为缺省字体大小(缺省选项)。
% \item[answers]     在每一个问题后附上答案。
% \item[marginline]  放置装订线。
% \item[16kpaper]    使用16开纸张(缺省使用A4纸张)。
% \item[noindent]    没有缩进。
% \item[printbox]    显示评分框。
% \end{description}
%
% \subsection{变量}
% \label{sec:Variable}
% \changes{v0.2}{2011/07/27}{试卷中改用英文标点符号}
% 本宏包在题量和分值等方面均以高考试卷为模板,
% 默认的变量值可以在\texttt{BHCexam.cfg}中设置,当然
% 你也可以在使用相关命令之前使用以下命令进行更改。\\\\
% \DescribeMacro{\biaoti}
% 设置标题信息。
% \begin{quote}
% |\biaoti|\marg*{TEXT}
% \end{quote}
% \DescribeMacro{\fubiaoti}
% 设置副标题,他会显示在标题下方。
% \begin{quote}     
% |\kemu|\marg*{TEXT}
% \end{quote}
% \DescribeMacro{\xinxi}
% 设置总分和考试时间信息,\parg{num1}为总分,\parg{num2}为考试时间。
% \begin{quote}
% |\xinxi|\marg*{num1}\marg*{num2}
% \end{quote}
% \DescribeMacro{\settk}
% 设置填空题的总分、题量和小分信息,\parg{num1}为总分,\parg{num2}为题量,
% \parg{num3}为小分。
% \begin{quote}
% |\settk|\marg*{num1}\marg*{num2}\marg*{num3}
% \end{quote}
% \DescribeMacro{\setxz}
% 设置选择题的总分、题量和小分信息,\parg{num1}为总分,\parg{num2}为题量,
% \parg{num3}为小分。
% \begin{quote}
% |\setxz|\marg*{num1}\marg*{num2}\marg*{num3}
% \end{quote}
% \DescribeMacro{\setjd}
% 设置简答题的总分、题量和小分信息,\parg{num1}为总分,\parg{num2}为题量。
% \begin{quote}
% |\setjd|\marg*{num1}\marg*{num2}\marg*{num3}
% \end{quote}

% \subsection{环境}
% \label{sec:Environment} 
% 经常使用的环境有\verb"questions"环境、\verb"parts"环境,
% 关于它们的简单介绍,请参考\ref{sec:Example},这里做一点补充说明,
% 更详细的介绍,请参考~\exam~文档。\\\\
% 在排版简答题时需要用\verb"\part"命令输入各小问的分值,宏包会自动算出总分并显示在该简答题的第一行。
% 当该道简答题没有小问时,则要用\verb"\question"命令输入该问题的分值。
% 
% \begin{verbatim}
%  ...
%  \jianda
%  \question 这是一道简答题
%  \begin{parts}[
%  \part[3] 第1小问3分。
%  \part[3] 第2小问3分。
%  \part[3] 第3小问4分。
%  \end{parts}
%  \question[12] 这是一道没有小问的简答题,这道题有12分
%  ... 
% \end{verbatim}
% 在\verb"\question"后新建\verb"solution"环境,在其中输入该问题的解答,
% 在不显示答案的情况下,该问题后会预留答题空间。
% \begin{verbatim}
%  ...
%  \jianda
%  \question 这是一道简答题
%  \begin{solution}
%  这是解答,不显示答案的情况下则这个问题后预留答题空间。
%  \end{solution}
%  ... 
% \end{verbatim}
%
% \subsection{常用命令}
% 
% \DescribeMacro{\newpage}
% 每道问题的间距是弹性设置的,你只要在想换页的地方输入\verb"\newpage"命令,
% 则上一页的各问题间距会自动调整到最美观的效果。\\
% \DescribeMacro{\mininotice}
% 在一行内输出精简的考试注意事项。\\
% \DescribeMacro{\printmalol}
% 在当前页为正面时,在左边插入装订线(仅在使用marginline选项时有效)。\\
% \DescribeMacro{\printmalol}
% 在当前页为反面时,在左边插入装订线(仅在使用marginline选项时有效)。\\
%
% \StopEventually{}
%
% \section{源代码说明}
%
% \subsection{选项}
%
%
% 处理~\BHCexam~文档类的选项
%
% \begin{macro}{\@sixteenkpaper}
% 16k纸张大小设置,缺省选项为a4paper
%    \begin{macrocode}
%<*cls>
\newif\if@sixteenkpaper \@sixteenkpaperfalse
\DeclareOption{16kpaper}{\@sixteenkpapertrue}
%</cls>
%    \end{macrocode}
% \end{macro}
%
% \begin{macro}{\@marginline}
% 是否有装订线
%    \begin{macrocode}
%<*cls>
\newif\if@marginline \@marginlinefalse
\DeclareOption{marginline}{\@marginlinetrue}
%</cls>
%    \end{macrocode}
% \end{macro}
%
% 不缩进,缺省为缩进
%    \begin{macrocode}
%<*cls>
\newif\if@noindent \@noindentfalse
\DeclareOption{noindent}{\@noindenttrue}
%</cls>
%    \end{macrocode}
%
% 显示答案的方式,缺省不显示答案
%    \begin{macrocode}
%<cls>\DeclareOption{answers}{\PassOptionsToClass{\CurrentOption}{exam}}
%    \end{macrocode}

% \begin{macro}{\@printbox}
% 显示计分框,缺省为不显示。
%    \begin{macrocode}
%<*cls>
\newif\if@printbox \@printboxfalse
\DeclareOption{printbox}{\@printboxtrue}
%</cls>
%    \end{macrocode}
% \end{macro}
%
% 把没有定义的选项传递给缺省的文档类
%    \begin{macrocode}
%<cls>\DeclareOption*{\PassOptionsToClass{\CurrentOption}{exam}}
%    \end{macrocode}
%
% 处理选项
%    \begin{macrocode}
%<cls>\ProcessOptions
%    \end{macrocode}
%
% 装入缺省的文档类
%    \begin{macrocode}
%<cls>\LoadClass[addpoints]{exam}
%    \end{macrocode}
% 
% 导入ctex类的实现
% \changes{v0.4}{2015/10/10}{修正ctex宏包实现}
%    \begin{macrocode}
%<*cls>
\RequirePackage{ctex}
%</cls>
%    \end{macrocode}
% \subsection{宏包}
%
% \begin{macro}{\RequirePackage}
% 我们需要使用的一些宏包
%    \begin{macrocode}
%<*cls>
\RequirePackage{ifpdf,ifxetex}
\RequirePackage{amsmath,amssymb,amsthm,bm,bbding,pifont,dsfont}
\RequirePackage{mathtools}
\RequirePackage{paralist,cases,tabularx}
\RequirePackage{pstricks,pst-plot,xcolor,graphicx}
%</cls>
%    \end{macrocode}
% 
% 用geometry宏包进行页面设置
% \changes{v0.2}{2011/07/27}{改用geometry宏包实现纸张设置}
% \changes{v0.3}{2014/02/18}{修正了纸张的尺寸}
%    \begin{macrocode}
%<*cls>
\if@marginline
\marginparwidth = 2cm
\if@sixteenkpaper
\RequirePackage[papersize={184mm,260mm},hmargin={3cm,2cm},
vmargin={2cm,2cm},marginparsep=0.5cm,hoffset=0cm,voffset=0cm,
footnotesep=0.5cm,headsep=0.5cm,twoside]{geometry}
\else
\RequirePackage[paper=a4paper,hmargin={3cm,2cm},vmargin={2cm,2cm},
marginparsep=0.5cm,hoffset=0cm,voffset=0cm,footnotesep=0.5cm,
headsep=0.5cm,twoside]{geometry}
\fi
\else
\if@sixteenkpaper
\RequirePackage[papersize={184mm,260mm},hmargin={2cm,2cm},
vmargin={2cm,2cm},marginparsep=0.5cm,hoffset=0cm,voffset=0cm,
footnotesep=0.5cm,headsep=0.5cm]{geometry}
\else
\RequirePackage[papersize={210mm,297mm},hmargin={2cm,2cm},vmargin={2cm,2cm},
marginparsep=0.5cm,hoffset=0cm,voffset=0cm,footnotesep=0.5cm,
headsep=0.5cm]{geometry}
\fi
\fi
%</cls>
%    \end{macrocode}
%\end{macro}
%
% \subsection{自定义设置}
%行距、页眉、页脚
%    \begin{macrocode}
%<*cls>
\renewcommand{\baselinestretch}{1.5}
\pagestyle{headandfoot}
\header{}{}{}
\footer{}{\small \quad 第~\thepage~页(共~\numpages~页)}{}
%</cls>
%    \end{macrocode}
%
% 分值显示
%    \begin{macrocode}
%<*cls>
\pointname{分}
\pointformat{\kaishu (\thepoints)}
%</cls>
%    \end{macrocode}
%
% 问题的显示
%    \begin{macrocode}
%<*cls>
\renewcommand{\questionshook}{
  \settowidth{\leftmargin}{22.\hskip\labelsep}
  \if@noindent \setlength\leftmargin{0pt} \fi
}
\renewcommand{\thepartno}{\arabic{partno}}
\renewcommand{\partlabel}{(\thepartno)}
\renewcommand{\partshook}{
  \settowidth{\leftmargin}{(3).\hskip\labelsep}
  \if@noindent \setlength\leftmargin{0pt} \fi
}
%</cls>
%    \end{macrocode}
%
% 解答的显示
%    \begin{macrocode}
%<*cls>
\newif\if@cancelspace \@cancelspacetrue
\renewcommand{\solutiontitle}{\noindent 解:\noindent}
\renewenvironment{solution}%
  {%
    \ifprintanswers
	%\unskip
      \begingroup
      \Solution@Emphasis
      \begin{TheSolution}%
    \else
      \if@cancelspace
        %\unskip
      \else
        \par
        \penalty 0
        \vfill%
        \if@printbox \if@houpinfen \houpinfen \fi \fi
      \fi
      \setbox\z@\vbox\bgroup
    \fi
  }{%
    \ifprintanswers
      \end{TheSolution}%
      \endgroup
    \else
      \egroup
    \fi
  }%
%</cls>
%    \end{macrocode}
%
% \subsection{新的命令和环境}
%
% \begin{macro}{\printmlor}
% \begin{macro}{\printmlol}
% \changes{v0.1}{2011/07/23}{支持在首页插入装订线}
% \changes{v0.2}{2011/07/27}{手动在指定页插入左(右)装订线}
% 装订线
%    \begin{macrocode}
%<*cls>
\if@marginline
\newsavebox{\zdxl}
\sbox{\zdxl}{
\begin{minipage}{0.7\paperheight}
\begin{center}
\heiti 班级\underline{\hspace{15ex}} \quad
姓名 \underline{\hspace{15ex}} \quad
学号 \underline{\hspace{15ex}} \quad \\
\vspace{3ex}
\dotfill 装 \dotfill 订 \dotfill 线 \dotfill
\end{center}
\end{minipage}
}
\newsavebox{\zdxr}
\sbox{\zdxr}{
\begin{minipage}{0.7\paperheight}
\begin{center}
\heiti \hfill 请 \hfill 不 \hfill 要 \hfill 在 \hfill
 装 \hfill 订 \hfill 线 \hfill 内 \hfill 答 \hfill 题 \hfill \\
\vspace{3ex}
\dotfill 装 \dotfill 订 \dotfill 线 \dotfill
\end{center}
\end{minipage}
}
\newcommand{\printmlol}{
\marginpar{\rotatebox{90}{\usebox{\zdxl}}}
}
\newcommand{\printmlor}{
\marginpar{\rotatebox{-90}{\usebox{\zdxr}}}
}
\reversemarginpar
\fi
%</cls>
%    \end{macrocode}
% \end{macro}
% \end{macro}
%
% \begin{macro}{\biaoti}
% \begin{macro}{\fubiaoti}
% 标题
%    \begin{macrocode}
%<*cls>
\newcommand\biaoti[1]{\def\@biaoti{#1}}
\newcommand\fubiaoti[1]{\def\@fubiaoti{#1}}
\renewcommand\maketitle{
  \begin{center}{\heiti \Large{\@biaoti}}\end{center}
  \begin{center}{\heiti \Large{\@fubiaoti}}\end{center}
}
%</cls>
%    \end{macrocode}
% \end{macro}
% \end{macro}
%
% \begin{macro}{\mininotice}
% \begin{macro}{\xinxi}
% 一行内显示考试时间和考试总分
%    \begin{macrocode}
%<*cls>
\newcommand\xinxi[2]{
  \def\@zongfen{#1}
  \def\@shijian{#2}
}
\newcommand\mininotice{
  \begin{center}{
    \kaishu (本试卷满分~\@zongfen~分, 考试时间~\@shijian~分钟)}
  \end{center}
}
%</cls>
%    \end{macrocode}
% \end{macro}
% \end{macro}
%
% \begin{macro}{\notice}
% 注意事项
%    \begin{macrocode}
%<*cls>
\newcommand{\notice}{
  \heiti 注意事项: \songti
  \begin{enumerate}
  \item 答卷前, 考生务必将姓名、高考准考证号、校验码等填写清楚.
  \item 本试卷共~\numquestions{}~道试题, 满分~\@zongfen~分,考试时间~\@shijian~分钟.
  \end{enumerate}
}
%</cls>
%    \end{macrocode}
%\end{macro}
%
% \begin{macro}{\pingfen}
% 前评分框
%    \begin{macrocode}
%<*cls>
\newlength\@boxwidth
\setlength\@boxwidth{0ex}
\if@printbox \setlength\@boxwidth{18ex} \fi
\newcommand\pinfen{
  \heiti
  \begin{minipage}{\@boxwidth}
  \begin{tabular}{|c|c|}
  \hline
  得分 & 评卷人\\
  \hline
      &       \\
  \hline
  \end{tabular}
  \end{minipage}
}
%</cls>
%    \end{macrocode}
% \end{macro}
%
% \begin{macro}{\houpinfen}
% 后评分框
%    \begin{macrocode}
%<*cls>
\newcommand{\houpinfen}{
  \hfill
  \begin{tabular}{|l|l|}
    \hline
    得分 & \hspace*{1.5cm}\\
    \hline
  \end{tabular}
  \bigskip
}
%</cls>
%    \end{macrocode}
% \end{macro}
% 
%
% 一些参数及变量
%    \begin{macrocode}
%<*cfg>
\def\@biaoti{2011年全国普通高等学校招生统一考试(上海卷)}
\def\@fubiaoti{数学(理科)}
\def\@zongfen{150}
\def\@shijian{120}
\def\@tiankong@zongfen{56}
\def\@tiankong@tishu{14}
\def\@tiankong@fen{4}
\def\@xuanze@zongfen{16}
\def\@xuanze@tishu{4}
\def\@xuanze@fen{4}
\def\@jianda@zongfen{78}
\def\@jianda@tishu{5}
%</cfg>
%<*cls>
\newcounter{@dati}
\newif\if@houpinfen \@houpinfenfalse
%</cls>
%    \end{macrocode}
%
% 填空题提示语
% \begin{macro}{\tiankong}
%    \begin{macrocode}
%<*cls>
\newcommand\settk[3]{
  \def\@tiankong@zongfen{#1}
  \def\@tiankong@tishu{#2}
  \def\@tiankong@fen{#3}
}
\newcommand\tiankong{
  \@houpinfenfalse
  \stepcounter{@dati}
  \fullwidth{
    \if@printbox \pinfen \fi
    \begin{minipage}{\textwidth-\@boxwidth}
    \heiti \chinese{@dati}. 填空题:本大题共~\@tiankong@tishu~题,满分~\@tiankong@zongfen~分. 请在横线上方填写最终的、最准确的、最完整的结果. 每题填写正确得~\@tiankong@fen~分,否则一律得0分.
    \end{minipage}
  }
}
%</cls>
%    \end{macrocode}
% \end{macro}
%
% \begin{macro}{\stk}
% \begin{macro}{\mtk}
% \begin{macro}{\ltk}
% \changes{v0.3}{2014/02/18}{设置三种长度的横线,并支持在横线上输出答案}
% 填空题横线
%    \begin{macrocode}
%<*cls>
\newcommand{\stk}[1]{
  \ifprintanswers
    \underline{~#1~}
  \else
    \underline{~\hspace{1cm}~}
  \fi}
\newcommand{\mtk}[1]{
  \ifprintanswers
    \underline{~#1~}
  \else
    \underline{~\hspace{2cm}~}
  \fi}
\newcommand{\ltk}[1]{
  \ifprintanswers
    \underline{~#1~}
  \else
    \underline{~\hspace{3cm}~}
  \fi}
%</cls>
%    \end{macrocode}
% \end{macro}
% \end{macro}
% \end{macro}
%
% 选择题提示语
% \begin{macro}{\xuanze}
%    \begin{macrocode}
%<*cls>
\newcommand\setxz[3]{
  \def\@xuanze@zongfen{#1}
  \def\@xuanze@tishu{#2}
  \def\@xuanze@fen{#3}
}
\newcommand\xuanze{
  \@houpinfenfalse
  \stepcounter{@dati}
  \fullwidth{
    \if@printbox \pinfen \fi
    \begin{minipage}{\textwidth-\@boxwidth}
      \heiti \chinese{@dati}. 选择题:本大题共~\@xuanze@tishu~题,满分~\@xuanze@zongfen~分. 请选择你认为最正确的答案(每小题有且只有一个)写在括号内. 每题填写正确得~\@xuanze@fen~分,否则得0分.
    \end{minipage}
  }
}
%</cls>
%    \end{macrocode}
% \end{macro}
%
% \begin{macro}{\onech}
% \begin{macro}{\twoch}
% \begin{macro}{\fourch}
% \changes{v0.3}{2014/02/18}{分一行、两行、四行显示选项}
% 选择题的选项
%    \begin{macrocode}
%<*cls>
\newcommand{\onech}[4]{
\renewcommand\arraystretch{1.4}
\begin{tabularx}{\linewidth}{XXXX}
\setlength\tabcolsep{0pt}
(A) #1 & (B) #2 & (C) #3 & (D) #4 \\
\end{tabularx}
\unskip \unskip}
\newcommand{\twoch}[4]{
\renewcommand\arraystretch{1.4}
\begin{tabularx}{\linewidth}{XX}
\setlength\tabcolsep{0pt}
(A) #1 & (B) #2 \\
(C) #3 & (D) #4
\end{tabularx}
\unskip \unskip}
\newcommand{\fourch}[4]{
\renewcommand\arraystretch{1.4}
\begin{tabularx}{\linewidth}{X}
\setlength\tabcolsep{0pt}
(A) #1 \\
(B) #2 \\
(C) #3 \\
(D) #4 \\
\end{tabularx}
\unskip \unskip}
%</cls>
%    \end{macrocode}
% \end{macro}
% \end{macro}
% \end{macro}
%
% 简答题提示语
% \begin{macro}{\jianda}
%    \begin{macrocode}
%<*cls>
\newcommand\setjd[2]{
  \def\@jianda@zongfen{#1}
  \def\@jianda@tishu{#2}
}
\newcommand\jianda{
  \@cancelspacefalse
  \@houpinfentrue
  \stepcounter{@dati}
  \fullwidth{
    \if@printbox \pinfen \fi
    \begin{minipage}{\textwidth-\@boxwidth}
      \heiti \chinese{@dati}. 简答题:本大题共~\@jianda@tishu~题,满分~\@jianda@zongfen~分. 请在题后空处写出必要的推理计算过程.
    \end{minipage}
  }
}
%</cls>
%    \end{macrocode}
% \end{macro}
%
% 数学运算符号、单位
%    \begin{macrocode}
%<*cls>
\delimitershortfall-1sp
\newcommand\abs[1]{\left|#1\right|}
\renewcommand{\geq}{\geqslant}
\renewcommand{\ge}{\geqslant}
\renewcommand{\leq}{\leqslant}
\renewcommand{\le}{\leqslant}
%<\cls>
%    \end{macrocode}
%
% 读取配置文件
%    \begin{macrocode}
%<*cls>
\AtEndOfPackage{\makeatletter% \iffalse meta-comment
%
% Copyright (C) 2011 by Charles Bao <charley792@gmail.com>
%
% This file is part of the BHCexam package project.
% ---------------------------------------------------
%
% It may be distributed under the conditions of the LaTeX Project Public
% License, either version 1.2 of this license or (at your option) any
% later version. The latest version of this license is in
%    http://www.latex-project.org/lppl.txt
% and version 1.2 or later is part of all distributions of LaTeX
% version 1999/12/01 or later.
%
%<*!(cfg|fd)>
% \fi
%
%% \CharacterTable
%%  {Upper-case    \A\B\C\D\E\F\G\H\I\J\K\L\M\N\O\P\Q\R\S\T\U\V\W\X\Y\Z
%%   Lower-case    \a\b\c\d\e\f\g\h\i\j\k\l\m\n\o\p\q\r\s\t\u\v\w\x\y\z
%%   Digits        \0\1\2\3\4\5\6\7\8\9
%%   Exclamation   \!     Double quote  \"     Hash (number) \#
%%   Dollar        \$     Percent       \%     Ampersand     \&
%%   Acute accent  \'     Left paren    \(     Right paren   \)
%%   Asterisk      \*     Plus          \+     Comma         \,
%%   Minus         \-     Point         \.     Solidus       \/
%%   Colon         \:     Semicolon     \;     Less than     \<
%%   Equals        \=     Greater than  \>     Question mark \?
%%   Commercial at \@     Left bracket  \[     Backslash     \\
%%   Right bracket \]     Circumflex    \^     Underscore    \_
%%   Grave accent  \`     Left brace    \{     Vertical bar  \|
%%   Right brace   \}     Tilde         \~}
%%
%
% \CheckSum{0}
%
% \iffalse meta-comment
%</!(cfg|fd)>
%
%<*driver>
\ProvidesFile{BHCexam.dtx}
%</driver>
%
%<cls>\NeedsTeXFormat{LaTeX2e}[1995/12/01]
%<cls>\ProvidesClass{BHCexam}
%<cfg>\ProvidesFile{BHCexam.cfg}
  [2011/07/27 v0.2 BHCexam
%<cls>   document class]
%<cfg>   configuration file]
%
%<*driver>
   bundle source file]
%</driver>
%
%<*driver>
\documentclass[a4paper]{ltxdoc}
\usepackage{ctex}
\usepackage{hyperref}
\usepackage{amsmath,amssymb}
 \topmargin 0.5 true cm
 \oddsidemargin 1 true cm
 \evensidemargin 1 true cm
 \textheight 21 true cm
 \textwidth 14 true cm
\EnableCrossrefs
 %\DisableCrossrefs % Say \DisableCrossrefs if index is ready
\CodelineIndex
\RecordChanges      % Gather update information
 %\OnlyDescription  % comment out for implementation details
 %\OldMakeindex     % use if your MakeIndex is pre-v2.9
\hypersetup{colorlinks,linkcolor=blue,citecolor=blue}
\begin{document}  
 \DocInput{BHCexam.dtx}
\end{document}
%</driver>
%
% \fi
%
%
% \changes{v0.0}{2011/07/22}{Initial version}
% \changes{v0.1}{2011/07/23}{version 0.1}
% \changes{v0.2}{2011/07/27}{version 0.2}
%
%
% \DoNotIndex{\begin,\end,\begingroup,\endgroup}
% \DoNotIndex{\ifx,\ifdim,\ifnum,\ifcase,\else,\or,\fi}
% \DoNotIndex{\let,\def,\xdef,\newcommand,\renewcommand}
% \DoNotIndex{\expandafter,\csname,\endcsname,\relax,\protect}
% \DoNotIndex{\Huge,\huge,\LARGE,\Large,\large,\normalsize}
% \DoNotIndex{\small,\footnotesize,\scriptsize,\tiny}
% \DoNotIndex{\normalfont,\bfseries,\slshape,\interlinepenalty}
% \DoNotIndex{\hfil,\par,\vskip,\vspace,\quad}
% \DoNotIndex{\centering,\raggedright}
% \DoNotIndex{\c@secnumdepth,\@startsection,\@setfontsize}
% \DoNotIndex{\ ,\@plus,\@minus,\p@,\z@,\@m,\@M,\@ne,\m@ne}
% \DoNotIndex{\@@par}
%
%
% \GetFileInfo{BHCexam.dtx}
%
%
% \MakeShortVerb{\|}
% \setcounter{StandardModuleDepth}{1}
%
%
% \newcommand{\ctex}{\texttt{ctex}}
% \newcommand{\BHCexam}{\texttt{BHCexam}}
% \newcommand{\exam}{\texttt{exam}}
% \newcommand{\colin}{\texttt{colinexam}}
%
%
% \setlength{\parskip}{0.75ex plus .2ex minus .5ex}
% \renewcommand{\baselinestretch}{1.2}
%
% \newcommand{\rc}{\text{C}}
% \newcommand{\ri}{\text{i}}
% \newcommand{\ra}{\text{A}}
% \newcommand{\rd}{\text{d}}
% \renewcommand\m{\ensuremath{\text{m}}}
% \newcommand\tian{\ensuremath{\text{d}}}
% \newcommand\A{\ensuremath{\text{A}}}
% \newcommand\g{\ensuremath{\text{g}}}
% \newcommand\kg{\ensuremath{\text{kg}}}
% \newcommand\degree{\ensuremath{^\circ}}
% \newcommand\ssd{\ensuremath{\text{\textcelsius}}}
% \newcommand\rad{\ensuremath{\text{rad}}}
% \newcommand\N{\ensuremath{\text{N}}}
% \newcommand\Pa{\ensuremath{\text{Pa}}}
% \newcommand\J{\ensuremath{\text{J}}}
% \newcommand\W{\ensuremath{\text{W}}}
% \newcommand\ohm{\ensuremath{\Omega}}
% \newcommand\mol{\ensuremath{\text{mol}}}
% \newcommand\K{\ensuremath{\text{K}}}
% \newcommand\h{\ensuremath{\text{h}}}
% \newcommand\ton{\ensuremath{\text{t}}}
% \newcommand\squarem{\ensuremath{\text{m$^2$}}}
% \newcommand\cubicm{\ensuremath{\text{m$^3$}}}
% \newcommand\cm{\ensuremath{\text{cm}}}
% \newcommand\mm{\ensuremath{\text{mm}}}
% \newcommand\squarecm{\ensuremath{\text{cm$^2$}}}
% \newcommand\cubiccm{\ensuremath{\text{cm$^3$}}}
% \newcommand\squaremm{\ensuremath{\text{mm$^2$}}}
% \newcommand\cubicmm{\ensuremath{\text{mm$^3$}}}
% \newcommand\liter{\ensuremath{\text{L}}}
% \newcommand{\abs}[1]{\left|#1\right|}
% \newcommand\arccot{\mathop{\text{arccot}}}
%
% \makeatletter
% \def\parg#1{\mbox{$\langle${\it #1\/}$\rangle$}}
% \def\@smarg#1{{\tt\string{}\parg{#1}{\tt\string}}}
% \def\@marg#1{{\tt\string{}{\rm #1}{\tt\string}}}
% \def\marg{\@ifstar\@smarg\@marg}
% \def\@soarg#1{{\tt[}\parg{#1}{\tt]}}
% \def\@oarg#1{{\tt[}{\rm #1}{\tt]}}
% \def\oarg{\@ifstar\@soarg\@oarg}
% \makeatother
%
%
% \title{\bf \BHCexam~宏包说明\thanks
%   {这是鲍宏昌发布的第一个~\LaTeX~宏包。本文件版本号为~\fileversion{},最后修改日期~\filedate{}。}}
% \author{\it 鲍宏昌\thanks{charley792@gmail.com}}
% \date{\small 打印日期:~\today}
% \maketitle
%
% \begin{abstract}
% \BHCexam~宏包提供了一个中学试卷排版的~\LaTeX{}~文档类。
%
% \BHCexam~主要文件包括~\texttt{BHCexam.cls}~文档类和配置文件
% ~\texttt{BHCexam.cfg}。
%
% \BHCexam~宏包由鲍宏昌制作并负责维护。
% \end{abstract}
%
% \tableofcontents
% \newpage
%
% \section{简介}
%
% 本宏包以~\exam~为底层文档类,部分源代码来自于盖鹤麟开发的
% ~\colin。不知道什么原因盖鹤麟自2004年就一直没有发布更新,
% ~\colin~仍然使用CCT实现中文支持,而缺乏对~XeTeX~的支持。2011年7月,
% 鲍宏昌在~\colin~的基础上改用~\ctex~实现中文支持,采用UTF8编码同时
% 支持~XeLaTeX~和~pdfTeX~进行编译,并使用~\texttt{doc}~和
% ~\texttt{docstrip}~工具编写了这个文档,增加了一些新的功能,
% 并把新的宏包命名为~\BHCexam。
%
% 本宏包延续了~\colin~和~\exam~的特点,能让一个刚刚接触
% ~\LaTeX~的初学者,也能轻松用它来排版试卷。希望~\BHCexam~能提高中学
% 教师的工作效率,并把注意力放在试卷的内容上。
% 
% \BHCexam~由两个主要文件构成:文档类~\texttt{BHCexam.cls}~和配置文件
% ~\texttt{BHCexam.cfg}~。后者定义了一些常用的参数。
%
% {\kaishu
% 这两个文件可以通过用~XeLaTeX~编译~\texttt{BHCexam.ins}~文件来得到,
% 而这份说明文档可以通过用~XeLaTeX~编译~\texttt{BHCexam.dtx}~文件来得到。
% 编译说明文档需要~\ctex{}~宏包,为了生成正确的索引和版本记录,
% 需要使用如下命令}
% \begin{verbatim}
% makeindex -s gind.ist -o BHCexam.ind BHCexam.idx
% makeindex -s gglo.ist -o BHCexam.gls BHCexam.glo
% \end{verbatim}
%
% \section{一个简单的例子}
%
% 用~\BHCexam~要排版一张基本的试卷其实很简单。如果你准备对试卷的排版进行
% 更细致的设置,那么请参考~\exam~的文档。
%
% \subsection{\texttt{documentclass} 命令}
% \label{sec:BasicDocumentclass}
%
% 要使用~\BHCexam~文档类,你的\verb"\documentclass" 命令应该是
% \begin{verbatim}
%  \documentclass{BHCexam}
% \end{verbatim}
% 如果,你想使用小四字体作为缺省字体大小,那么添加选项\verb"cs4size"
% \begin{verbatim}
% \documentclass[cs4size]{BHCexam}
% \end{verbatim}
% 更多的选项,请参考\ref{sec:Options}。
%
% \subsection{打印标题和考试须知}
% \DescribeMacro{\maketitle}
% \DescribeMacro{\notice}
% 在试卷上打印标题和考试须知
% \begin{verbatim}
%   \maketitle
%   \notice 
% \end{verbatim}
% 关于标题和考试须知中变量的设置,请参考\ref{sec:Variable}。
%
% \subsection{题目}
% \label{sec:Example}
% \DescribeMacro{\question}
% \DescribeMacro{\choice}
% \DescribeMacro{\part}
% \DescribeMacro{\tiankong}
% \DescribeMacro{\xuanze}
% \DescribeMacro{\jianda}
% 在\verb"questions"环境中用\verb"\question"输入题目。
% 在\verb"choices"环境中用\verb"\choice"输入选项。
% 在\verb"parts"环境中用\verb"\part"输入小问。
% 在\verb"questions"环境中用\verb"\tiankong"、\verb"\xuanze"
% 和\verb"\jianda"分别显示填空题、选择题、简答题的提示语。
% \begin{verbatim}
% \begin{questions}
%   \tiankong
%   \question 这是第1道填空题
%   \question 这是第2道填空题
%   \xuanze
%   \question 问题3是一道选择题
%   \begin{choices}
%     \choice 选项1
%     \choice 选项2
%     \choice 选项3
%     \choice 选项4
%   \begin{choices}
%   \jianda
%   \question 问题4是一道简答题
%   \begin{parts}
%   \part 第1小问
%   \part 第2小问
%   \end{parts}
% \end{questions}
% \end{verbatim}
% 关于填空题、选择题、简答题的提示语中几个变量的设置,请参考\ref{sec:Variable}。
% 关于题目的更多内容,请参考\ref{sec:Environment}。
%
% \section{使用帮助}
%
% \subsection{选项}
% \label{sec:Options}
%
% \changes{v0.2}{2011/07/27}{增加UTF8选项以支持pdflatex}
%
% 宏包的选项用于改变一些缺省的设置。虽然缺省的设置基本能过满足一般用户的
% 使用需要,但用户也可以根据自己的情况,使用这些选项。
%
% \begin{description}
% \item[cs4size]     使用小四字号为缺省字体大小。
% \item[c5size]      使用五号字为缺省字体大小(缺省选项)。
% \item[answers]     在每一个问题后附上答案。
% \item[marginline]  放置装订线。
% \item[16kpaper]    使用16开纸张(缺省使用A4纸张)。
% \item[noindent]    没有缩进。
% \item[printbox]    显示评分框。
% \item[UTF8]        用pdflatex编译时需要。
% \end{description}
%
% \subsection{变量}
% \label{sec:Variable}
% \changes{v0.2}{2011/07/27}{试卷中改用英文标点符号}
% 本宏包在题量和分值等方面均以高考试卷为模板,
% 默认的变量值可以在\texttt{BHCexam.cfg}中设置,当然
% 你也可以在使用相关命令之前使用以下命令进行更改。\\\\
% \DescribeMacro{\biaoti}
% 设置标题信息。
% \begin{quote}
% |\biaoti|\marg*{TEXT}
% \end{quote}
% \DescribeMacro{\kemu}
% 设置科目信息,他会显示在标题下方和页脚内。
% \begin{quote}     
% |\kemu|\marg*{TEXT}
% \end{quote}
% \DescribeMacro{\xinxi}
% 设置总分和考试时间信息,\parg{num1}为总分,\parg{num2}为考试时间。
% \begin{quote}
% |\xinxi|\marg*{num1}\marg*{num2}
% \end{quote}
% \DescribeMacro{\settk}
% 设置填空题的总分、题量和小分信息,\parg{num1}为总分,\parg{num2}为题量,
% \parg{num3}为小分。
% \begin{quote}
% |\settk|\marg*{num1}\marg*{num2}\marg*{num3}
% \end{quote}
% \DescribeMacro{\setxz}
% 设置选择题的总分、题量和小分信息,\parg{num1}为总分,\parg{num2}为题量,
% \parg{num3}为小分。
% \begin{quote}
% |\setxz|\marg*{num1}\marg*{num2}\marg*{num3}
% \end{quote}
% \DescribeMacro{\setjd}
% 设置简答题的总分、题量和小分信息,\parg{num1}为总分,\parg{num2}为题量。
% \begin{quote}
% |\setjd|\marg*{num1}\marg*{num2}\marg*{num3}
% \end{quote}

% \subsection{环境}
% \label{sec:Environment} 
% 经常使用的环境有\verb"questions"环境、\verb"choices"环境、\verb"parts"环境,
% 关于它们的简单介绍,请参考\ref{sec:Example},这里做一点补充说明,
% 更详细的介绍,请参考~\exam~文档。\\\\
% 在排版简答题时需要用\verb"\part"命令输入各小问的分值,宏包会自动算出总分并显示在该简答题的第一行。
% 当该道简答题没有小问时,则要用\verb"\question"命令输入该问题的分值。
% 
% \begin{verbatim}
%  ...
%  \jianda
%  \question 这是一道简答题
%  \begin{parts}[
%  \part[3] 第1小问3分。
%  \part[3] 第2小问3分。
%  \part[3] 第3小问4分。
%  \end{parts}
%  \question[12] 这是一道没有小问的简答题,这道题有12分
%  ... 
% \end{verbatim}
% 在\verb"\question"后新建\verb"solution"环境,在其中输入该问题的解答,
% 在不显示答案的情况下,该问题后会预留答题空间。
% \begin{verbatim}
%  ...
%  \jianda
%  \question 这是一道简答题
%  \begin{solution}
%  这是解答,不显示答案的情况下则这个问题后预留答题空间。
%  \end{solution}
%  ... 
% \end{verbatim}
%
% \subsection{常用命令}
% 
% \DescribeMacro{\newpage}
% 每道问题的间距是弹性设置的,你只要在想换页的地方输入\verb"\newpage"命令,
% 则上一页的各问题间距会自动调整到最美观的效果。\\
% \DescribeMacro{\mininotice}
% 在一行内输出精简的考试注意事项。\\
% \DescribeMacro{\printmalol}
% 在当前页为正面时,在左边插入装订线(仅在使用marginline选项时有效)。\\
% \DescribeMacro{\printmalol}
% 在当前页为反面时,在左边插入装订线(仅在使用marginline选项时有效)。\\

% \subsection{数学符号}
% \begin{tabular}{ll}
% \hline
% \verb"\oneb" & 生成一条很小的供排版填空题空格用的横线\\\hline
% \verb"\twob" & 生成一条长一些的供排版填空题空格用的横线\\\hline
% \verb"\fourb" & 生成一长更长的供排版填空题空格用的横线\\\hline
% \verb"\sixb" & 生成一长较长的供排版填空题空格用的横线\\\hline
% \verb"\tenb" & 生成一条很长的供排版填空题空格用的横线\\\hline
% \verb"\rc" & 在数学模式下输入正体的~C(组合数符号)\\\hline
% \verb"\ra" & 在数学模式下输入正体的~A(排列数符号)\\\hline
% \verb"\ri" & 在数学模式下输入正体的~i(虚数单位)\\\hline
% \verb"\abs{...}" & 用以输入绝对值\\\hline
% \verb"\arccot" & 在数学模式下输入$\arccot$\\\hline
% \verb"\m" & 在文本模式和数学模式下均可使用,生成单位符号~\m\\\hline
% \verb"\W" & 在文本模式攻数学模式下均可使用,生成单位符号~\W\\\hline
% \verb"\A" & 在文本模式和数学模式下均可使用,生成单位符号~\A\\ \hline
% \verb"\g" & 在文本模式和数学模式下均可使用,生成单位符号~\g\\ \hline
% \verb"\kg" & 在文本模式和数学模式下均可使用,生成单位符号~\kg\\ \hline
% \verb"\degree" & 在文本模式和数学模式下均可使用,生成单位符号~\degree\\\hline
% \verb"\ssd" & 在文本模式和数学模式下均可使用,生成单位符号~\ssd\\\hline
% \verb"\rad" & 在文本模式和数学模式下均可使用,生成单位符号~\rad\\\hline
% \verb"\N" & 在文本模式和数学模式下均可使用,生成单位符号~\N\\\hline
% \verb"\Pa" & 在文本模式和数学模式下均可使用,生成单位符号~\Pa\\\hline
% \verb"\J" & 在文本模式和数学模式下均可使用,生成单位符号~\J\\\hline
% \verb"\ohm" & 在文本模式和数学模式下均可使用,生成单位符号~\ohm\\\hline
% \verb"\mol" & 在文本模式和数学模式下均可使用,生成单位符号~\mol\\\hline
% \verb"\K" & 在文本模式和数学模式下均可使用,生成单位符号~\K\\\hline
% \verb"\h" & 在文本模式和数学模式下均可使用,生成单位符号~\h\\\hline
% \verb"\ton" & 在文本模式和数学模式下均可使用,生成单位符号~\ton\\\hline
% \verb"\squarem" & 在文本模式和数学模式下均可使用,生成单位符号~\squarem\\\hline
% \verb"\cubicm" & 在文本模式和数学模式下均可使用,生成单位符号~\cubicm\\\hline
% \verb"\cm" & 在文本模式和数学模式下均可使用,生成单位符号~\cm\\\hline
% \verb"\mm" & 在文本模式和数学模式下均可使用,生成单位符号~\mm\\\hline
% \verb"\squarecm" & 在文本模式和数学模式下均可使用,生成单位符号~\squarecm\\\hline
% \verb"\cubiccm" & 在文本模式和数学模式下均可使用,生成单位符号~\cubiccm\\\hline
% \verb"\squaremm" & 在文本模式和数学模式下均可使用,生成单位符号~\squaremm\\\hline
% \verb"\cubicmm" & 在文本模式和数学模式下均可使用,生成单位符号~\cubicmm\\\hline
% \verb"\liter" & 在文本模式和数学模式下均可使用,生成单位符号~\liter\\\hline
% \end{tabular}
%
% \StopEventually{
% } ^^A end StopEventually
%
% \section{源代码说明}
%
% \subsection{选项}
%
%\begin{macro}{\input}
% \changes{v0.2}{2011/07/27}{改用input语句导入ctex类文件}
% 导入ctex类的选项
%    \begin{macrocode}
%<*cls>
\input{ctex-common-opts.def}
\input{ctex-class-opts.def}
\input{ctex-caption-opts.def}
%</cls>
%    \end{macrocode}
%\end{macro}
%
% 处理~\BHCexam~文档类的选项
%
% \begin{macro}{\@sixteenkpaper}
% 16k纸张大小设置,缺省选项为a4paper
%    \begin{macrocode}
%<*cls>
\newif\if@sixteenkpaper \@sixteenkpapertrue
\DeclareOption{16kpaper}{\@sixteenkpaperfalse}
%</cls>
%    \end{macrocode}
% \end{macro}
%
% \begin{macro}{\@marginline}
% 是否有装订线
%    \begin{macrocode}
%<*cls>
\newif\if@marginline \@marginlinefalse
\DeclareOption{marginline}{\@marginlinetrue}
%</cls>
%    \end{macrocode}
% \end{macro}
%
% 不缩进,缺省为缩进
%    \begin{macrocode}
%<*cls>
\newif\if@noindent \@noindentfalse
\DeclareOption{noindent}{\@noindenttrue}
%</cls>
%    \end{macrocode}
%
% 显示答案的方式,缺省不显示答案
%    \begin{macrocode}
%<cls>\DeclareOption{answers}{\PassOptionsToClass{\CurrentOption}{exam}}
%    \end{macrocode}

% \begin{macro}{\@printbox}
% 显示计分框,缺省为不显示。
%    \begin{macrocode}
%<*cls>
\newif\if@printbox \@printboxfalse
\DeclareOption{printbox}{\@printboxtrue}
%</cls>
%    \end{macrocode}
% \end{macro}
%
%
% 把没有定义的选项传递给缺省的文档类
%    \begin{macrocode}
%<cls>\DeclareOption*{\PassOptionsToClass{\CurrentOption}{exam}}
%    \end{macrocode}
%
% 处理选项
%    \begin{macrocode}
%<cls>\ProcessOptions
%    \end{macrocode}
%
% 装入缺省的文档类
%    \begin{macrocode}
%<cls>\LoadClass[addpoints]{exam}
%    \end{macrocode}
% 
% 导入ctex类的实现
%    \begin{macrocode}
%<*cls>
\input{ctex-common.def}
\input{ctex-caption.def}
\input{ctex-class.def}
%</cls>
%    \end{macrocode}
% \subsection{宏包}
%
% \begin{macro}{\RequirePackage}
% 我们需要使用的一些宏包
%    \begin{macrocode}
%<*cls>
\RequirePackage{amsmath,amssymb,bm}
\RequirePackage{graphicx}
\RequirePackage{ifpdf,ifxetex}
%</cls>
%    \end{macrocode}
% 
% 用geometry宏包进行页面设置
% \changes{v0.2}{2011/07/27}{改用geometry宏包实现纸张设置}
%    \begin{macrocode}
%<*cls>
\if@marginline
\if@sixteenkpaper
\RequirePackage[papersize={184mm,260mm},hmargin={3cm,2cm},
vmargin={2cm,2cm},marginparsep=0.5cm,hoffset=0cm,voffset=0cm,
footnotesep=0.5cm,headsep=0.5cm,twoside]{geometry}
\else
\RequirePackage[paper=a4paper,hmargin={3cm,2cm},vmargin={2cm,2cm},
marginparsep=0.5cm,hoffset=0cm,voffset=0cm,footnotesep=0.5cm,
headsep=0.5cm,twoside]{geometry}
\fi
\else
\if@sixteenkpaper
\RequirePackage[papersize={184mm,260mm},hmargin={2cm,2cm},
vmargin={2cm,2cm},marginparsep=0.5cm,hoffset=0cm,voffset=0cm,
footnotesep=0.5cm,headsep=0.5cm]{geometry}
\else
\RequirePackage[paper=a4paper,hmargin={2cm,2cm},vmargin={2cm,2cm},
marginparsep=0.5cm,hoffset=0cm,voffset=0cm,footnotesep=0.5cm,
headsep=0.5cm]{geometry}
\fi
\fi
%</cls>
%    \end{macrocode}
%\end{macro}
%
% \subsection{自定义设置}
%行距、页眉、页脚
%    \begin{macrocode}
%<*cls>
\renewcommand{\baselinestretch}{1.5}
\pagestyle{headandfoot}
%\runningheadrule
%\firstpageheadrule
%\runningfootrule
%\firstpagefootrule
\header{}{}{}
\footer{}{\small \kaishu{\@kemu}\quad 第~\thepage~页(共~\numpages~页)}{}
%</cls>
%    \end{macrocode}
%
% 分值显示
%    \begin{macrocode}
%<*cls>
\pointname{分}
\pointformat{\kaishu (\thepoints)}
%</cls>
%    \end{macrocode}
%
% 解的显示
%    \begin{macrocode}
%<*cls>
\renewcommand{\solutiontitle}{\noindent\heiti{解:}\noindent}
%</cls>
%    \end{macrocode}
%
% 小问的显示
%    \begin{macrocode}
%<*cls>
\renewcommand{\thepartno}{\arabic{partno}}
\renewcommand{\partlabel}{(\thepartno)}
\renewcommand{\partshook}{
  \settowidth{\leftmargin}{(3).\hskip\labelsep}
  \if@noindent \setlength\leftmargin{0pt} \fi
}
%</cls>
%    \end{macrocode}
%
% 选项的显示
%    \begin{macrocode}
%<*cls>
\renewcommand{\choiceshook}{
  \settowidth{\leftmargin}{W.\hskip\labelsep}
  \if@noindent \setlength\leftmargin{0pt} \fi
}
%</cls>
%    \end{macrocode}
%
% 解答的显示
%    \begin{macrocode}
%<*cls>
\renewenvironment{solution}%
  {%
    \ifprintanswers
      \begingroup
      \Solution@Emphasis
      \begin{TheSolution}%
    \else
      \ifcancelspace
        % Do nothing
      \else
        \par
        \penalty 0
        \vfill%
        \if@printbox \if@houpinfen \houpinfen \fi \fi
      \fi
      \setbox\z@\vbox\bgroup
    \fi
  }{%
    \ifprintanswers
      \end{TheSolution}%
      \endgroup
    \else
      \egroup
    \fi
  }%
%</cls>
%    \end{macrocode}
%
% \subsection{新的命令和环境}
%
% \begin{macro}{\printmlor}
% \begin{macro}{\printmlol}
% \changes{v0.1}{2011/07/23}{支持在首页插入装订线}
% \changes{v0.2}{2011/07/27}{手动在指定页插入左(右)装订线}
% 装订线
%    \begin{macrocode}
%<*cls>
\if@marginline 
\newsavebox{\zdxl}
\sbox{\zdxl}{
\begin{minipage}{0.7\paperheight} 
\begin{center}
\heiti 班级\underline{\hspace{15ex}} \quad
姓名 \underline{\hspace{15ex}} \quad
学号 \underline{\hspace{15ex}} \quad \\ 
\vspace{3ex}
\dotfill 装 \dotfill 订 \dotfill 线 \dotfill 
\end{center}
\end{minipage}
}
\newsavebox{\zdxr}
\sbox{\zdxr}{
\begin{minipage}{0.7\paperheight} 
\begin{center}
\heiti \hfill 请 \hfill 不 \hfill 要 \hfill 在 \hfill
 装 \hfill 订 \hfill 线 \hfill 内 \hfill 答 \hfill 题 \hfill \\ 
\vspace{3ex}
\dotfill 装 \dotfill 订 \dotfill 线 \dotfill 
\end{center}
\end{minipage}
}
\newcommand{\printmlol}{
\marginpar{\rotatebox{90}{\usebox{\zdxl}}}
}
\newcommand{\printmlor}{
\marginpar{\rotatebox{-90}{\usebox{\zdxr}}}
}
\reversemarginpar
\fi
%</cls>
%    \end{macrocode}
% \end{macro}
% \end{macro}
%
% \begin{macro}{\biaoti}
% \begin{macro}{\kemu}
% 标题
%    \begin{macrocode}
%<*cfg>
\def\@biaoti{2011年全国普通高等学校招生统一考试(上海卷)}
\def\@kemu{数学(理科)}
%</cfg>
%<*cls>
\newcommand\biaoti[1]{\def\@biaoti{#1}}
\newcommand\kemu[1]{\def\@kemu{#1}}
\renewcommand\maketitle{
  \begin{center}{\heiti \Large{\@biaoti}}\end{center}
  \begin{center}{\heiti \Large{\@kemu}}\end{center}
}
%</cls>
%    \end{macrocode}
% \end{macro}
% \end{macro}
%
% \begin{macro}{\mininotice}
% \begin{macro}{\xinxi}
% 一行内显示考试时间和考试总分
%    \begin{macrocode}{\mininotice}
%<*cfg>
\def\@zongfen{150}
\def\@shijian{120}
%</cfg>
%<*cls>
\newcommand\xinxi[2]{
  \def\@zongfen{#1}
  \def\@shijian{#2}
}
\newcommand\mininotice{
  \begin{center}{
    \kaishu (本试卷满分~\@zongfen~分, 考试时间~\@shijian~分钟)}
  \end{center}
}
%</cls>
%    \end{macrocode}
% \end{macro}
% \end{macro}
%
% \begin{macro}{\notice}
% 注意事项
%    \begin{macrocode}
%<*cls>
\newcommand{\notice}{
  \heiti 注意事项: \songti
  \begin{enumerate}
  \item 答卷前, 考生务必将姓名、高考准考证号、校验码等填写清楚.
  \item 本试卷共~\numquestions{}~道试题, 满分~\@zongfen~分,考试时间~\@shijian~分钟.
  \end{enumerate}
}
%</cls>
%    \end{macrocode}
%\end{macro}
%
% \begin{macro}{\pingfen}
% 前评分框
%    \begin{macrocode}
%<*cls>
\newlength\@boxwidth
\setlength\@boxwidth{0ex}
\if@printbox \setlength\@boxwidth{18ex} \fi
\newcommand\pinfen{
  \heiti
  \begin{minipage}{\@boxwidth}
  \begin{tabular}{|c|c|}
  \hline
  得分 & 评卷人\\
  \hline
      &       \\ 
  \hline
  \end{tabular}
  \end{minipage}
}
%</cls>
%    \end{macrocode}
% \end{macro}
%
% \begin{macro}{\houpinfen}
% 后评分框
%    \begin{macrocode}
%<*cls>
\newcommand{\houpinfen}{
  \hfill
  \begin{tabular}{|l|l|}
    \hline
    得分 & \hspace*{1.5cm}\\
    \hline
  \end{tabular}
  \bigskip
}
%</cls>
%    \end{macrocode}
% \end{macro}
%
% \begin{macro}{\oneb}
% \begin{macro}{\twob}
% \begin{macro}{\sixb}
% \begin{macro}{\tenb}
% 空格
%    \begin{macrocode}
%<*cls>
\newcommand\oneb{\underline{\hspace{1em}}\hspace{0.001em}}
\newcommand\twob{\oneb\oneb}
\newcommand{\sixb}{\twob\twob}
\newcommand\tenb{\twob\twob\twob\twob\twob}
%</cls>
%    \end{macrocode}
% \end{macro}
% \end{macro}
% \end{macro}
% \end{macro}
%
% 填空题、选择题、简答题
%    \begin{macrocode}
%<*cfg>
\def\@tiankong@zongfen{56}
\def\@tiankong@tishu{14}
\def\@tiankong@fen{4}
\def\@xuanze@zongfen{16}
\def\@xuanze@tishu{4}
\def\@xuanze@fen{4}
\def\@jianda@zongfen{78}
\def\@jianda@tishu{5}
%</cfg>
%<*cls>
\newcounter{@dati}
\newif\if@houpinfen \@houpinfenfalse
\newcommand\settk[3]{
  \def\@tiankong@zongfen{#1}
  \def\@tiankong@tishu{#2}
  \def\@tiankong@fen{#3}
}
\newcommand\tiankong{
  \@houpinfenfalse
  \stepcounter{@dati} 
  \fullwidth{
    \if@printbox \pinfen \fi
    \begin{minipage}{\textwidth-\@boxwidth}
    \heiti \chinese{@dati}. 填空题(\kaishu 本大题满分~\@tiankong@zongfen~分) \heiti 本大题有~\@tiankong@tishu~题, 考生应在答题纸相应编号的空格内直接写结果, 每个空格填对得~\@tiankong@fen~分, 否则一律得零分.
    \end{minipage}
  }
}
\newcommand\setxz[3]{
  \def\@xuanze@zongfen{#1}
  \def\@xuanze@tishu{#2}
  \def\@xuanze@fen{#3}
}
\newcommand\xuanze{
  \@houpinfenfalse
  \stepcounter{@dati} 
  \fullwidth{
    \if@printbox \pinfen \fi
    \begin{minipage}{\textwidth-\@boxwidth}  
      \heiti \chinese{@dati}. 选择题(\kaishu 本大题满分~\@xuanze@zongfen~分) \heiti 本大题共有~\@xuanze@tishu~题, 每题有且只有一个正确答案, 考生应在答题纸的相应编号上, 将代表答案的小方格涂黑, 选对得~\@xuanze@fen~分, 否则一律得零分.
    \end{minipage}
  }
}
\newcommand\setjd[2]{
  \def\@jianda@zongfen{#1}
  \def\@jianda@tishu{#2}
}
\newcommand\jianda{
  \@houpinfentrue
  \qformat{\hskip\labelsep \kaishu \thequestion.~~(本题满分~\totalpoints~分)\hfill}
  \stepcounter{@dati}
  \fullwidth{
    \if@printbox \pinfen \fi  
    \begin{minipage}{\textwidth-\@boxwidth}
      \heiti \chinese{@dati}. 简答题(\kaishu 本大题满分~\@jianda@zongfen~分)~\heiti 本大题共有~\@jianda@tishu~题, 解答下列各题必须在答题纸相应的编号规定区域内写出必要的步骤.
    \end{minipage}
  }
}
%</cls>
%    \end{macrocode}
%
% 数学运算符号、单位
%    \begin{macrocode}
%<*cls>
\newcommand{\rc}{\text{C}}
\newcommand{\ri}{\text{i}}
\newcommand{\ra}{\text{A}}
\newcommand{\rd}{\text{d}}
\newcommand\tian{\ensuremath{\text{d}}}
\newcommand\A{\ensuremath{\text{A}}}
\def\m{\ensuremath{\text{m}}}
\newcommand\g{\ensuremath{\text{g}}}
\newcommand\kg{\ensuremath{\text{kg}}}
\newcommand\degree{\ensuremath{^\circ}}
\newcommand\ssd{\ensuremath{\text{\textcelsius}}}
\newcommand\rad{\ensuremath{\text{rad}}}
\newcommand\N{\ensuremath{\text{N}}}
\newcommand\Pa{\ensuremath{\text{Pa}}}
\newcommand\J{\ensuremath{\text{J}}}
\newcommand\W{\ensuremath{\text{W}}}
\newcommand\ohm{\ensuremath{\Omega}}
\newcommand\mol{\ensuremath{\text{mol}}}
\newcommand\K{\ensuremath{\text{K}}}
\newcommand\h{\ensuremath{\text{h}}}
\newcommand\ton{\ensuremath{\text{t}}}
\newcommand\squarem{\ensuremath{\text{m$^2$}}}
\newcommand\cubicm{\ensuremath{\text{m$^3$}}}
\newcommand\cm{\ensuremath{\text{cm}}}
\newcommand\mm{\ensuremath{\text{mm}}}
\newcommand\squarecm{\ensuremath{\text{cm$^2$}}}
\newcommand\cubiccm{\ensuremath{\text{cm$^3$}}}
\newcommand\squaremm{\ensuremath{\text{mm$^2$}}}
\newcommand\cubicmm{\ensuremath{\text{mm$^3$}}}
\newcommand\liter{\ensuremath{\text{L}}}
\newcommand{\abs}[1]{\left|#1\right|}
\newcommand\arccot{\mathop{\text{arccot}}}
\newcommand\pingxing{\parallel}
%<\cls>
%    \end{macrocode}
%
% 读取配置文件
%    \begin{macrocode}
%<*cls>
\AtEndOfPackage{\makeatletter% \iffalse meta-comment
%
% Copyright (C) 2011 by Charles Bao <charley792@gmail.com>
%
% This file is part of the BHCexam package project.
% ---------------------------------------------------
%
% It may be distributed under the conditions of the LaTeX Project Public
% License, either version 1.2 of this license or (at your option) any
% later version. The latest version of this license is in
%    http://www.latex-project.org/lppl.txt
% and version 1.2 or later is part of all distributions of LaTeX
% version 1999/12/01 or later.
%
%<*!(cfg|fd)>
% \fi
%
%% \CharacterTable
%%  {Upper-case    \A\B\C\D\E\F\G\H\I\J\K\L\M\N\O\P\Q\R\S\T\U\V\W\X\Y\Z
%%   Lower-case    \a\b\c\d\e\f\g\h\i\j\k\l\m\n\o\p\q\r\s\t\u\v\w\x\y\z
%%   Digits        \0\1\2\3\4\5\6\7\8\9
%%   Exclamation   \!     Double quote  \"     Hash (number) \#
%%   Dollar        \$     Percent       \%     Ampersand     \&
%%   Acute accent  \'     Left paren    \(     Right paren   \)
%%   Asterisk      \*     Plus          \+     Comma         \,
%%   Minus         \-     Point         \.     Solidus       \/
%%   Colon         \:     Semicolon     \;     Less than     \<
%%   Equals        \=     Greater than  \>     Question mark \?
%%   Commercial at \@     Left bracket  \[     Backslash     \\
%%   Right bracket \]     Circumflex    \^     Underscore    \_
%%   Grave accent  \`     Left brace    \{     Vertical bar  \|
%%   Right brace   \}     Tilde         \~}
%%
%
% \CheckSum{0}
%
% \iffalse meta-comment
%</!(cfg|fd)>
%
%<*driver>
\ProvidesFile{BHCexam.dtx}
%</driver>
%
%<cls>\NeedsTeXFormat{LaTeX2e}[1995/12/01]
%<cls>\ProvidesClass{BHCexam}
%<cfg>\ProvidesFile{BHCexam.cfg}
  [2011/07/27 v0.2 BHCexam
%<cls>   document class]
%<cfg>   configuration file]
%
%<*driver>
   bundle source file]
%</driver>
%
%<*driver>
\documentclass[a4paper]{ltxdoc}
\usepackage{ctex}
\usepackage{hyperref}
\usepackage{amsmath,amssymb}
 \topmargin 0.5 true cm
 \oddsidemargin 1 true cm
 \evensidemargin 1 true cm
 \textheight 21 true cm
 \textwidth 14 true cm
\EnableCrossrefs
 %\DisableCrossrefs % Say \DisableCrossrefs if index is ready
\CodelineIndex
\RecordChanges      % Gather update information
 %\OnlyDescription  % comment out for implementation details
 %\OldMakeindex     % use if your MakeIndex is pre-v2.9
\hypersetup{colorlinks,linkcolor=blue,citecolor=blue}
\begin{document}  
 \DocInput{BHCexam.dtx}
\end{document}
%</driver>
%
% \fi
%
%
% \changes{v0.0}{2011/07/22}{Initial version}
% \changes{v0.1}{2011/07/23}{version 0.1}
% \changes{v0.2}{2011/07/27}{version 0.2}
%
%
% \DoNotIndex{\begin,\end,\begingroup,\endgroup}
% \DoNotIndex{\ifx,\ifdim,\ifnum,\ifcase,\else,\or,\fi}
% \DoNotIndex{\let,\def,\xdef,\newcommand,\renewcommand}
% \DoNotIndex{\expandafter,\csname,\endcsname,\relax,\protect}
% \DoNotIndex{\Huge,\huge,\LARGE,\Large,\large,\normalsize}
% \DoNotIndex{\small,\footnotesize,\scriptsize,\tiny}
% \DoNotIndex{\normalfont,\bfseries,\slshape,\interlinepenalty}
% \DoNotIndex{\hfil,\par,\vskip,\vspace,\quad}
% \DoNotIndex{\centering,\raggedright}
% \DoNotIndex{\c@secnumdepth,\@startsection,\@setfontsize}
% \DoNotIndex{\ ,\@plus,\@minus,\p@,\z@,\@m,\@M,\@ne,\m@ne}
% \DoNotIndex{\@@par}
%
%
% \GetFileInfo{BHCexam.dtx}
%
%
% \MakeShortVerb{\|}
% \setcounter{StandardModuleDepth}{1}
%
%
% \newcommand{\ctex}{\texttt{ctex}}
% \newcommand{\BHCexam}{\texttt{BHCexam}}
% \newcommand{\exam}{\texttt{exam}}
% \newcommand{\colin}{\texttt{colinexam}}
%
%
% \setlength{\parskip}{0.75ex plus .2ex minus .5ex}
% \renewcommand{\baselinestretch}{1.2}
%
% \newcommand{\rc}{\text{C}}
% \newcommand{\ri}{\text{i}}
% \newcommand{\ra}{\text{A}}
% \newcommand{\rd}{\text{d}}
% \renewcommand\m{\ensuremath{\text{m}}}
% \newcommand\tian{\ensuremath{\text{d}}}
% \newcommand\A{\ensuremath{\text{A}}}
% \newcommand\g{\ensuremath{\text{g}}}
% \newcommand\kg{\ensuremath{\text{kg}}}
% \newcommand\degree{\ensuremath{^\circ}}
% \newcommand\ssd{\ensuremath{\text{\textcelsius}}}
% \newcommand\rad{\ensuremath{\text{rad}}}
% \newcommand\N{\ensuremath{\text{N}}}
% \newcommand\Pa{\ensuremath{\text{Pa}}}
% \newcommand\J{\ensuremath{\text{J}}}
% \newcommand\W{\ensuremath{\text{W}}}
% \newcommand\ohm{\ensuremath{\Omega}}
% \newcommand\mol{\ensuremath{\text{mol}}}
% \newcommand\K{\ensuremath{\text{K}}}
% \newcommand\h{\ensuremath{\text{h}}}
% \newcommand\ton{\ensuremath{\text{t}}}
% \newcommand\squarem{\ensuremath{\text{m$^2$}}}
% \newcommand\cubicm{\ensuremath{\text{m$^3$}}}
% \newcommand\cm{\ensuremath{\text{cm}}}
% \newcommand\mm{\ensuremath{\text{mm}}}
% \newcommand\squarecm{\ensuremath{\text{cm$^2$}}}
% \newcommand\cubiccm{\ensuremath{\text{cm$^3$}}}
% \newcommand\squaremm{\ensuremath{\text{mm$^2$}}}
% \newcommand\cubicmm{\ensuremath{\text{mm$^3$}}}
% \newcommand\liter{\ensuremath{\text{L}}}
% \newcommand{\abs}[1]{\left|#1\right|}
% \newcommand\arccot{\mathop{\text{arccot}}}
%
% \makeatletter
% \def\parg#1{\mbox{$\langle${\it #1\/}$\rangle$}}
% \def\@smarg#1{{\tt\string{}\parg{#1}{\tt\string}}}
% \def\@marg#1{{\tt\string{}{\rm #1}{\tt\string}}}
% \def\marg{\@ifstar\@smarg\@marg}
% \def\@soarg#1{{\tt[}\parg{#1}{\tt]}}
% \def\@oarg#1{{\tt[}{\rm #1}{\tt]}}
% \def\oarg{\@ifstar\@soarg\@oarg}
% \makeatother
%
%
% \title{\bf \BHCexam~宏包说明\thanks
%   {这是鲍宏昌发布的第一个~\LaTeX~宏包。本文件版本号为~\fileversion{},最后修改日期~\filedate{}。}}
% \author{\it 鲍宏昌\thanks{charley792@gmail.com}}
% \date{\small 打印日期:~\today}
% \maketitle
%
% \begin{abstract}
% \BHCexam~宏包提供了一个中学试卷排版的~\LaTeX{}~文档类。
%
% \BHCexam~主要文件包括~\texttt{BHCexam.cls}~文档类和配置文件
% ~\texttt{BHCexam.cfg}。
%
% \BHCexam~宏包由鲍宏昌制作并负责维护。
% \end{abstract}
%
% \tableofcontents
% \newpage
%
% \section{简介}
%
% 本宏包以~\exam~为底层文档类,部分源代码来自于盖鹤麟开发的
% ~\colin。不知道什么原因盖鹤麟自2004年就一直没有发布更新,
% ~\colin~仍然使用CCT实现中文支持,而缺乏对~XeTeX~的支持。2011年7月,
% 鲍宏昌在~\colin~的基础上改用~\ctex~实现中文支持,采用UTF8编码同时
% 支持~XeLaTeX~和~pdfTeX~进行编译,并使用~\texttt{doc}~和
% ~\texttt{docstrip}~工具编写了这个文档,增加了一些新的功能,
% 并把新的宏包命名为~\BHCexam。
%
% 本宏包延续了~\colin~和~\exam~的特点,能让一个刚刚接触
% ~\LaTeX~的初学者,也能轻松用它来排版试卷。希望~\BHCexam~能提高中学
% 教师的工作效率,并把注意力放在试卷的内容上。
% 
% \BHCexam~由两个主要文件构成:文档类~\texttt{BHCexam.cls}~和配置文件
% ~\texttt{BHCexam.cfg}~。后者定义了一些常用的参数。
%
% {\kaishu
% 这两个文件可以通过用~XeLaTeX~编译~\texttt{BHCexam.ins}~文件来得到,
% 而这份说明文档可以通过用~XeLaTeX~编译~\texttt{BHCexam.dtx}~文件来得到。
% 编译说明文档需要~\ctex{}~宏包,为了生成正确的索引和版本记录,
% 需要使用如下命令}
% \begin{verbatim}
% makeindex -s gind.ist -o BHCexam.ind BHCexam.idx
% makeindex -s gglo.ist -o BHCexam.gls BHCexam.glo
% \end{verbatim}
%
% \section{一个简单的例子}
%
% 用~\BHCexam~要排版一张基本的试卷其实很简单。如果你准备对试卷的排版进行
% 更细致的设置,那么请参考~\exam~的文档。
%
% \subsection{\texttt{documentclass} 命令}
% \label{sec:BasicDocumentclass}
%
% 要使用~\BHCexam~文档类,你的\verb"\documentclass" 命令应该是
% \begin{verbatim}
%  \documentclass{BHCexam}
% \end{verbatim}
% 如果,你想使用小四字体作为缺省字体大小,那么添加选项\verb"cs4size"
% \begin{verbatim}
% \documentclass[cs4size]{BHCexam}
% \end{verbatim}
% 更多的选项,请参考\ref{sec:Options}。
%
% \subsection{打印标题和考试须知}
% \DescribeMacro{\maketitle}
% \DescribeMacro{\notice}
% 在试卷上打印标题和考试须知
% \begin{verbatim}
%   \maketitle
%   \notice 
% \end{verbatim}
% 关于标题和考试须知中变量的设置,请参考\ref{sec:Variable}。
%
% \subsection{题目}
% \label{sec:Example}
% \DescribeMacro{\question}
% \DescribeMacro{\choice}
% \DescribeMacro{\part}
% \DescribeMacro{\tiankong}
% \DescribeMacro{\xuanze}
% \DescribeMacro{\jianda}
% 在\verb"questions"环境中用\verb"\question"输入题目。
% 在\verb"choices"环境中用\verb"\choice"输入选项。
% 在\verb"parts"环境中用\verb"\part"输入小问。
% 在\verb"questions"环境中用\verb"\tiankong"、\verb"\xuanze"
% 和\verb"\jianda"分别显示填空题、选择题、简答题的提示语。
% \begin{verbatim}
% \begin{questions}
%   \tiankong
%   \question 这是第1道填空题
%   \question 这是第2道填空题
%   \xuanze
%   \question 问题3是一道选择题
%   \begin{choices}
%     \choice 选项1
%     \choice 选项2
%     \choice 选项3
%     \choice 选项4
%   \begin{choices}
%   \jianda
%   \question 问题4是一道简答题
%   \begin{parts}
%   \part 第1小问
%   \part 第2小问
%   \end{parts}
% \end{questions}
% \end{verbatim}
% 关于填空题、选择题、简答题的提示语中几个变量的设置,请参考\ref{sec:Variable}。
% 关于题目的更多内容,请参考\ref{sec:Environment}。
%
% \section{使用帮助}
%
% \subsection{选项}
% \label{sec:Options}
%
% \changes{v0.2}{2011/07/27}{增加UTF8选项以支持pdflatex}
%
% 宏包的选项用于改变一些缺省的设置。虽然缺省的设置基本能过满足一般用户的
% 使用需要,但用户也可以根据自己的情况,使用这些选项。
%
% \begin{description}
% \item[cs4size]     使用小四字号为缺省字体大小。
% \item[c5size]      使用五号字为缺省字体大小(缺省选项)。
% \item[answers]     在每一个问题后附上答案。
% \item[marginline]  放置装订线。
% \item[16kpaper]    使用16开纸张(缺省使用A4纸张)。
% \item[noindent]    没有缩进。
% \item[printbox]    显示评分框。
% \item[UTF8]        用pdflatex编译时需要。
% \end{description}
%
% \subsection{变量}
% \label{sec:Variable}
% \changes{v0.2}{2011/07/27}{试卷中改用英文标点符号}
% 本宏包在题量和分值等方面均以高考试卷为模板,
% 默认的变量值可以在\texttt{BHCexam.cfg}中设置,当然
% 你也可以在使用相关命令之前使用以下命令进行更改。\\\\
% \DescribeMacro{\biaoti}
% 设置标题信息。
% \begin{quote}
% |\biaoti|\marg*{TEXT}
% \end{quote}
% \DescribeMacro{\kemu}
% 设置科目信息,他会显示在标题下方和页脚内。
% \begin{quote}     
% |\kemu|\marg*{TEXT}
% \end{quote}
% \DescribeMacro{\xinxi}
% 设置总分和考试时间信息,\parg{num1}为总分,\parg{num2}为考试时间。
% \begin{quote}
% |\xinxi|\marg*{num1}\marg*{num2}
% \end{quote}
% \DescribeMacro{\settk}
% 设置填空题的总分、题量和小分信息,\parg{num1}为总分,\parg{num2}为题量,
% \parg{num3}为小分。
% \begin{quote}
% |\settk|\marg*{num1}\marg*{num2}\marg*{num3}
% \end{quote}
% \DescribeMacro{\setxz}
% 设置选择题的总分、题量和小分信息,\parg{num1}为总分,\parg{num2}为题量,
% \parg{num3}为小分。
% \begin{quote}
% |\setxz|\marg*{num1}\marg*{num2}\marg*{num3}
% \end{quote}
% \DescribeMacro{\setjd}
% 设置简答题的总分、题量和小分信息,\parg{num1}为总分,\parg{num2}为题量。
% \begin{quote}
% |\setjd|\marg*{num1}\marg*{num2}\marg*{num3}
% \end{quote}

% \subsection{环境}
% \label{sec:Environment} 
% 经常使用的环境有\verb"questions"环境、\verb"choices"环境、\verb"parts"环境,
% 关于它们的简单介绍,请参考\ref{sec:Example},这里做一点补充说明,
% 更详细的介绍,请参考~\exam~文档。\\\\
% 在排版简答题时需要用\verb"\part"命令输入各小问的分值,宏包会自动算出总分并显示在该简答题的第一行。
% 当该道简答题没有小问时,则要用\verb"\question"命令输入该问题的分值。
% 
% \begin{verbatim}
%  ...
%  \jianda
%  \question 这是一道简答题
%  \begin{parts}[
%  \part[3] 第1小问3分。
%  \part[3] 第2小问3分。
%  \part[3] 第3小问4分。
%  \end{parts}
%  \question[12] 这是一道没有小问的简答题,这道题有12分
%  ... 
% \end{verbatim}
% 在\verb"\question"后新建\verb"solution"环境,在其中输入该问题的解答,
% 在不显示答案的情况下,该问题后会预留答题空间。
% \begin{verbatim}
%  ...
%  \jianda
%  \question 这是一道简答题
%  \begin{solution}
%  这是解答,不显示答案的情况下则这个问题后预留答题空间。
%  \end{solution}
%  ... 
% \end{verbatim}
%
% \subsection{常用命令}
% 
% \DescribeMacro{\newpage}
% 每道问题的间距是弹性设置的,你只要在想换页的地方输入\verb"\newpage"命令,
% 则上一页的各问题间距会自动调整到最美观的效果。\\
% \DescribeMacro{\mininotice}
% 在一行内输出精简的考试注意事项。\\
% \DescribeMacro{\printmalol}
% 在当前页为正面时,在左边插入装订线(仅在使用marginline选项时有效)。\\
% \DescribeMacro{\printmalol}
% 在当前页为反面时,在左边插入装订线(仅在使用marginline选项时有效)。\\

% \subsection{数学符号}
% \begin{tabular}{ll}
% \hline
% \verb"\oneb" & 生成一条很小的供排版填空题空格用的横线\\\hline
% \verb"\twob" & 生成一条长一些的供排版填空题空格用的横线\\\hline
% \verb"\fourb" & 生成一长更长的供排版填空题空格用的横线\\\hline
% \verb"\sixb" & 生成一长较长的供排版填空题空格用的横线\\\hline
% \verb"\tenb" & 生成一条很长的供排版填空题空格用的横线\\\hline
% \verb"\rc" & 在数学模式下输入正体的~C(组合数符号)\\\hline
% \verb"\ra" & 在数学模式下输入正体的~A(排列数符号)\\\hline
% \verb"\ri" & 在数学模式下输入正体的~i(虚数单位)\\\hline
% \verb"\abs{...}" & 用以输入绝对值\\\hline
% \verb"\arccot" & 在数学模式下输入$\arccot$\\\hline
% \verb"\m" & 在文本模式和数学模式下均可使用,生成单位符号~\m\\\hline
% \verb"\W" & 在文本模式攻数学模式下均可使用,生成单位符号~\W\\\hline
% \verb"\A" & 在文本模式和数学模式下均可使用,生成单位符号~\A\\ \hline
% \verb"\g" & 在文本模式和数学模式下均可使用,生成单位符号~\g\\ \hline
% \verb"\kg" & 在文本模式和数学模式下均可使用,生成单位符号~\kg\\ \hline
% \verb"\degree" & 在文本模式和数学模式下均可使用,生成单位符号~\degree\\\hline
% \verb"\ssd" & 在文本模式和数学模式下均可使用,生成单位符号~\ssd\\\hline
% \verb"\rad" & 在文本模式和数学模式下均可使用,生成单位符号~\rad\\\hline
% \verb"\N" & 在文本模式和数学模式下均可使用,生成单位符号~\N\\\hline
% \verb"\Pa" & 在文本模式和数学模式下均可使用,生成单位符号~\Pa\\\hline
% \verb"\J" & 在文本模式和数学模式下均可使用,生成单位符号~\J\\\hline
% \verb"\ohm" & 在文本模式和数学模式下均可使用,生成单位符号~\ohm\\\hline
% \verb"\mol" & 在文本模式和数学模式下均可使用,生成单位符号~\mol\\\hline
% \verb"\K" & 在文本模式和数学模式下均可使用,生成单位符号~\K\\\hline
% \verb"\h" & 在文本模式和数学模式下均可使用,生成单位符号~\h\\\hline
% \verb"\ton" & 在文本模式和数学模式下均可使用,生成单位符号~\ton\\\hline
% \verb"\squarem" & 在文本模式和数学模式下均可使用,生成单位符号~\squarem\\\hline
% \verb"\cubicm" & 在文本模式和数学模式下均可使用,生成单位符号~\cubicm\\\hline
% \verb"\cm" & 在文本模式和数学模式下均可使用,生成单位符号~\cm\\\hline
% \verb"\mm" & 在文本模式和数学模式下均可使用,生成单位符号~\mm\\\hline
% \verb"\squarecm" & 在文本模式和数学模式下均可使用,生成单位符号~\squarecm\\\hline
% \verb"\cubiccm" & 在文本模式和数学模式下均可使用,生成单位符号~\cubiccm\\\hline
% \verb"\squaremm" & 在文本模式和数学模式下均可使用,生成单位符号~\squaremm\\\hline
% \verb"\cubicmm" & 在文本模式和数学模式下均可使用,生成单位符号~\cubicmm\\\hline
% \verb"\liter" & 在文本模式和数学模式下均可使用,生成单位符号~\liter\\\hline
% \end{tabular}
%
% \StopEventually{
% } ^^A end StopEventually
%
% \section{源代码说明}
%
% \subsection{选项}
%
%\begin{macro}{\input}
% \changes{v0.2}{2011/07/27}{改用input语句导入ctex类文件}
% 导入ctex类的选项
%    \begin{macrocode}
%<*cls>
\input{ctex-common-opts.def}
\input{ctex-class-opts.def}
\input{ctex-caption-opts.def}
%</cls>
%    \end{macrocode}
%\end{macro}
%
% 处理~\BHCexam~文档类的选项
%
% \begin{macro}{\@sixteenkpaper}
% 16k纸张大小设置,缺省选项为a4paper
%    \begin{macrocode}
%<*cls>
\newif\if@sixteenkpaper \@sixteenkpapertrue
\DeclareOption{16kpaper}{\@sixteenkpaperfalse}
%</cls>
%    \end{macrocode}
% \end{macro}
%
% \begin{macro}{\@marginline}
% 是否有装订线
%    \begin{macrocode}
%<*cls>
\newif\if@marginline \@marginlinefalse
\DeclareOption{marginline}{\@marginlinetrue}
%</cls>
%    \end{macrocode}
% \end{macro}
%
% 不缩进,缺省为缩进
%    \begin{macrocode}
%<*cls>
\newif\if@noindent \@noindentfalse
\DeclareOption{noindent}{\@noindenttrue}
%</cls>
%    \end{macrocode}
%
% 显示答案的方式,缺省不显示答案
%    \begin{macrocode}
%<cls>\DeclareOption{answers}{\PassOptionsToClass{\CurrentOption}{exam}}
%    \end{macrocode}

% \begin{macro}{\@printbox}
% 显示计分框,缺省为不显示。
%    \begin{macrocode}
%<*cls>
\newif\if@printbox \@printboxfalse
\DeclareOption{printbox}{\@printboxtrue}
%</cls>
%    \end{macrocode}
% \end{macro}
%
%
% 把没有定义的选项传递给缺省的文档类
%    \begin{macrocode}
%<cls>\DeclareOption*{\PassOptionsToClass{\CurrentOption}{exam}}
%    \end{macrocode}
%
% 处理选项
%    \begin{macrocode}
%<cls>\ProcessOptions
%    \end{macrocode}
%
% 装入缺省的文档类
%    \begin{macrocode}
%<cls>\LoadClass[addpoints]{exam}
%    \end{macrocode}
% 
% 导入ctex类的实现
%    \begin{macrocode}
%<*cls>
\input{ctex-common.def}
\input{ctex-caption.def}
\input{ctex-class.def}
%</cls>
%    \end{macrocode}
% \subsection{宏包}
%
% \begin{macro}{\RequirePackage}
% 我们需要使用的一些宏包
%    \begin{macrocode}
%<*cls>
\RequirePackage{amsmath,amssymb,bm}
\RequirePackage{graphicx}
\RequirePackage{ifpdf,ifxetex}
%</cls>
%    \end{macrocode}
% 
% 用geometry宏包进行页面设置
% \changes{v0.2}{2011/07/27}{改用geometry宏包实现纸张设置}
%    \begin{macrocode}
%<*cls>
\if@marginline
\if@sixteenkpaper
\RequirePackage[papersize={184mm,260mm},hmargin={3cm,2cm},
vmargin={2cm,2cm},marginparsep=0.5cm,hoffset=0cm,voffset=0cm,
footnotesep=0.5cm,headsep=0.5cm,twoside]{geometry}
\else
\RequirePackage[paper=a4paper,hmargin={3cm,2cm},vmargin={2cm,2cm},
marginparsep=0.5cm,hoffset=0cm,voffset=0cm,footnotesep=0.5cm,
headsep=0.5cm,twoside]{geometry}
\fi
\else
\if@sixteenkpaper
\RequirePackage[papersize={184mm,260mm},hmargin={2cm,2cm},
vmargin={2cm,2cm},marginparsep=0.5cm,hoffset=0cm,voffset=0cm,
footnotesep=0.5cm,headsep=0.5cm]{geometry}
\else
\RequirePackage[paper=a4paper,hmargin={2cm,2cm},vmargin={2cm,2cm},
marginparsep=0.5cm,hoffset=0cm,voffset=0cm,footnotesep=0.5cm,
headsep=0.5cm]{geometry}
\fi
\fi
%</cls>
%    \end{macrocode}
%\end{macro}
%
% \subsection{自定义设置}
%行距、页眉、页脚
%    \begin{macrocode}
%<*cls>
\renewcommand{\baselinestretch}{1.5}
\pagestyle{headandfoot}
%\runningheadrule
%\firstpageheadrule
%\runningfootrule
%\firstpagefootrule
\header{}{}{}
\footer{}{\small \kaishu{\@kemu}\quad 第~\thepage~页(共~\numpages~页)}{}
%</cls>
%    \end{macrocode}
%
% 分值显示
%    \begin{macrocode}
%<*cls>
\pointname{分}
\pointformat{\kaishu (\thepoints)}
%</cls>
%    \end{macrocode}
%
% 解的显示
%    \begin{macrocode}
%<*cls>
\renewcommand{\solutiontitle}{\noindent\heiti{解:}\noindent}
%</cls>
%    \end{macrocode}
%
% 小问的显示
%    \begin{macrocode}
%<*cls>
\renewcommand{\thepartno}{\arabic{partno}}
\renewcommand{\partlabel}{(\thepartno)}
\renewcommand{\partshook}{
  \settowidth{\leftmargin}{(3).\hskip\labelsep}
  \if@noindent \setlength\leftmargin{0pt} \fi
}
%</cls>
%    \end{macrocode}
%
% 选项的显示
%    \begin{macrocode}
%<*cls>
\renewcommand{\choiceshook}{
  \settowidth{\leftmargin}{W.\hskip\labelsep}
  \if@noindent \setlength\leftmargin{0pt} \fi
}
%</cls>
%    \end{macrocode}
%
% 解答的显示
%    \begin{macrocode}
%<*cls>
\renewenvironment{solution}%
  {%
    \ifprintanswers
      \begingroup
      \Solution@Emphasis
      \begin{TheSolution}%
    \else
      \ifcancelspace
        % Do nothing
      \else
        \par
        \penalty 0
        \vfill%
        \if@printbox \if@houpinfen \houpinfen \fi \fi
      \fi
      \setbox\z@\vbox\bgroup
    \fi
  }{%
    \ifprintanswers
      \end{TheSolution}%
      \endgroup
    \else
      \egroup
    \fi
  }%
%</cls>
%    \end{macrocode}
%
% \subsection{新的命令和环境}
%
% \begin{macro}{\printmlor}
% \begin{macro}{\printmlol}
% \changes{v0.1}{2011/07/23}{支持在首页插入装订线}
% \changes{v0.2}{2011/07/27}{手动在指定页插入左(右)装订线}
% 装订线
%    \begin{macrocode}
%<*cls>
\if@marginline 
\newsavebox{\zdxl}
\sbox{\zdxl}{
\begin{minipage}{0.7\paperheight} 
\begin{center}
\heiti 班级\underline{\hspace{15ex}} \quad
姓名 \underline{\hspace{15ex}} \quad
学号 \underline{\hspace{15ex}} \quad \\ 
\vspace{3ex}
\dotfill 装 \dotfill 订 \dotfill 线 \dotfill 
\end{center}
\end{minipage}
}
\newsavebox{\zdxr}
\sbox{\zdxr}{
\begin{minipage}{0.7\paperheight} 
\begin{center}
\heiti \hfill 请 \hfill 不 \hfill 要 \hfill 在 \hfill
 装 \hfill 订 \hfill 线 \hfill 内 \hfill 答 \hfill 题 \hfill \\ 
\vspace{3ex}
\dotfill 装 \dotfill 订 \dotfill 线 \dotfill 
\end{center}
\end{minipage}
}
\newcommand{\printmlol}{
\marginpar{\rotatebox{90}{\usebox{\zdxl}}}
}
\newcommand{\printmlor}{
\marginpar{\rotatebox{-90}{\usebox{\zdxr}}}
}
\reversemarginpar
\fi
%</cls>
%    \end{macrocode}
% \end{macro}
% \end{macro}
%
% \begin{macro}{\biaoti}
% \begin{macro}{\kemu}
% 标题
%    \begin{macrocode}
%<*cfg>
\def\@biaoti{2011年全国普通高等学校招生统一考试(上海卷)}
\def\@kemu{数学(理科)}
%</cfg>
%<*cls>
\newcommand\biaoti[1]{\def\@biaoti{#1}}
\newcommand\kemu[1]{\def\@kemu{#1}}
\renewcommand\maketitle{
  \begin{center}{\heiti \Large{\@biaoti}}\end{center}
  \begin{center}{\heiti \Large{\@kemu}}\end{center}
}
%</cls>
%    \end{macrocode}
% \end{macro}
% \end{macro}
%
% \begin{macro}{\mininotice}
% \begin{macro}{\xinxi}
% 一行内显示考试时间和考试总分
%    \begin{macrocode}{\mininotice}
%<*cfg>
\def\@zongfen{150}
\def\@shijian{120}
%</cfg>
%<*cls>
\newcommand\xinxi[2]{
  \def\@zongfen{#1}
  \def\@shijian{#2}
}
\newcommand\mininotice{
  \begin{center}{
    \kaishu (本试卷满分~\@zongfen~分, 考试时间~\@shijian~分钟)}
  \end{center}
}
%</cls>
%    \end{macrocode}
% \end{macro}
% \end{macro}
%
% \begin{macro}{\notice}
% 注意事项
%    \begin{macrocode}
%<*cls>
\newcommand{\notice}{
  \heiti 注意事项: \songti
  \begin{enumerate}
  \item 答卷前, 考生务必将姓名、高考准考证号、校验码等填写清楚.
  \item 本试卷共~\numquestions{}~道试题, 满分~\@zongfen~分,考试时间~\@shijian~分钟.
  \end{enumerate}
}
%</cls>
%    \end{macrocode}
%\end{macro}
%
% \begin{macro}{\pingfen}
% 前评分框
%    \begin{macrocode}
%<*cls>
\newlength\@boxwidth
\setlength\@boxwidth{0ex}
\if@printbox \setlength\@boxwidth{18ex} \fi
\newcommand\pinfen{
  \heiti
  \begin{minipage}{\@boxwidth}
  \begin{tabular}{|c|c|}
  \hline
  得分 & 评卷人\\
  \hline
      &       \\ 
  \hline
  \end{tabular}
  \end{minipage}
}
%</cls>
%    \end{macrocode}
% \end{macro}
%
% \begin{macro}{\houpinfen}
% 后评分框
%    \begin{macrocode}
%<*cls>
\newcommand{\houpinfen}{
  \hfill
  \begin{tabular}{|l|l|}
    \hline
    得分 & \hspace*{1.5cm}\\
    \hline
  \end{tabular}
  \bigskip
}
%</cls>
%    \end{macrocode}
% \end{macro}
%
% \begin{macro}{\oneb}
% \begin{macro}{\twob}
% \begin{macro}{\sixb}
% \begin{macro}{\tenb}
% 空格
%    \begin{macrocode}
%<*cls>
\newcommand\oneb{\underline{\hspace{1em}}\hspace{0.001em}}
\newcommand\twob{\oneb\oneb}
\newcommand{\sixb}{\twob\twob}
\newcommand\tenb{\twob\twob\twob\twob\twob}
%</cls>
%    \end{macrocode}
% \end{macro}
% \end{macro}
% \end{macro}
% \end{macro}
%
% 填空题、选择题、简答题
%    \begin{macrocode}
%<*cfg>
\def\@tiankong@zongfen{56}
\def\@tiankong@tishu{14}
\def\@tiankong@fen{4}
\def\@xuanze@zongfen{16}
\def\@xuanze@tishu{4}
\def\@xuanze@fen{4}
\def\@jianda@zongfen{78}
\def\@jianda@tishu{5}
%</cfg>
%<*cls>
\newcounter{@dati}
\newif\if@houpinfen \@houpinfenfalse
\newcommand\settk[3]{
  \def\@tiankong@zongfen{#1}
  \def\@tiankong@tishu{#2}
  \def\@tiankong@fen{#3}
}
\newcommand\tiankong{
  \@houpinfenfalse
  \stepcounter{@dati} 
  \fullwidth{
    \if@printbox \pinfen \fi
    \begin{minipage}{\textwidth-\@boxwidth}
    \heiti \chinese{@dati}. 填空题(\kaishu 本大题满分~\@tiankong@zongfen~分) \heiti 本大题有~\@tiankong@tishu~题, 考生应在答题纸相应编号的空格内直接写结果, 每个空格填对得~\@tiankong@fen~分, 否则一律得零分.
    \end{minipage}
  }
}
\newcommand\setxz[3]{
  \def\@xuanze@zongfen{#1}
  \def\@xuanze@tishu{#2}
  \def\@xuanze@fen{#3}
}
\newcommand\xuanze{
  \@houpinfenfalse
  \stepcounter{@dati} 
  \fullwidth{
    \if@printbox \pinfen \fi
    \begin{minipage}{\textwidth-\@boxwidth}  
      \heiti \chinese{@dati}. 选择题(\kaishu 本大题满分~\@xuanze@zongfen~分) \heiti 本大题共有~\@xuanze@tishu~题, 每题有且只有一个正确答案, 考生应在答题纸的相应编号上, 将代表答案的小方格涂黑, 选对得~\@xuanze@fen~分, 否则一律得零分.
    \end{minipage}
  }
}
\newcommand\setjd[2]{
  \def\@jianda@zongfen{#1}
  \def\@jianda@tishu{#2}
}
\newcommand\jianda{
  \@houpinfentrue
  \qformat{\hskip\labelsep \kaishu \thequestion.~~(本题满分~\totalpoints~分)\hfill}
  \stepcounter{@dati}
  \fullwidth{
    \if@printbox \pinfen \fi  
    \begin{minipage}{\textwidth-\@boxwidth}
      \heiti \chinese{@dati}. 简答题(\kaishu 本大题满分~\@jianda@zongfen~分)~\heiti 本大题共有~\@jianda@tishu~题, 解答下列各题必须在答题纸相应的编号规定区域内写出必要的步骤.
    \end{minipage}
  }
}
%</cls>
%    \end{macrocode}
%
% 数学运算符号、单位
%    \begin{macrocode}
%<*cls>
\newcommand{\rc}{\text{C}}
\newcommand{\ri}{\text{i}}
\newcommand{\ra}{\text{A}}
\newcommand{\rd}{\text{d}}
\newcommand\tian{\ensuremath{\text{d}}}
\newcommand\A{\ensuremath{\text{A}}}
\def\m{\ensuremath{\text{m}}}
\newcommand\g{\ensuremath{\text{g}}}
\newcommand\kg{\ensuremath{\text{kg}}}
\newcommand\degree{\ensuremath{^\circ}}
\newcommand\ssd{\ensuremath{\text{\textcelsius}}}
\newcommand\rad{\ensuremath{\text{rad}}}
\newcommand\N{\ensuremath{\text{N}}}
\newcommand\Pa{\ensuremath{\text{Pa}}}
\newcommand\J{\ensuremath{\text{J}}}
\newcommand\W{\ensuremath{\text{W}}}
\newcommand\ohm{\ensuremath{\Omega}}
\newcommand\mol{\ensuremath{\text{mol}}}
\newcommand\K{\ensuremath{\text{K}}}
\newcommand\h{\ensuremath{\text{h}}}
\newcommand\ton{\ensuremath{\text{t}}}
\newcommand\squarem{\ensuremath{\text{m$^2$}}}
\newcommand\cubicm{\ensuremath{\text{m$^3$}}}
\newcommand\cm{\ensuremath{\text{cm}}}
\newcommand\mm{\ensuremath{\text{mm}}}
\newcommand\squarecm{\ensuremath{\text{cm$^2$}}}
\newcommand\cubiccm{\ensuremath{\text{cm$^3$}}}
\newcommand\squaremm{\ensuremath{\text{mm$^2$}}}
\newcommand\cubicmm{\ensuremath{\text{mm$^3$}}}
\newcommand\liter{\ensuremath{\text{L}}}
\newcommand{\abs}[1]{\left|#1\right|}
\newcommand\arccot{\mathop{\text{arccot}}}
\newcommand\pingxing{\parallel}
%<\cls>
%    \end{macrocode}
%
% 读取配置文件
%    \begin{macrocode}
%<*cls>
\AtEndOfPackage{\makeatletter% \iffalse meta-comment
%
% Copyright (C) 2011 by Charles Bao <charley792@gmail.com>
%
% This file is part of the BHCexam package project.
% ---------------------------------------------------
%
% It may be distributed under the conditions of the LaTeX Project Public
% License, either version 1.2 of this license or (at your option) any
% later version. The latest version of this license is in
%    http://www.latex-project.org/lppl.txt
% and version 1.2 or later is part of all distributions of LaTeX
% version 1999/12/01 or later.
%
%<*!(cfg|fd)>
% \fi
%
%% \CharacterTable
%%  {Upper-case    \A\B\C\D\E\F\G\H\I\J\K\L\M\N\O\P\Q\R\S\T\U\V\W\X\Y\Z
%%   Lower-case    \a\b\c\d\e\f\g\h\i\j\k\l\m\n\o\p\q\r\s\t\u\v\w\x\y\z
%%   Digits        \0\1\2\3\4\5\6\7\8\9
%%   Exclamation   \!     Double quote  \"     Hash (number) \#
%%   Dollar        \$     Percent       \%     Ampersand     \&
%%   Acute accent  \'     Left paren    \(     Right paren   \)
%%   Asterisk      \*     Plus          \+     Comma         \,
%%   Minus         \-     Point         \.     Solidus       \/
%%   Colon         \:     Semicolon     \;     Less than     \<
%%   Equals        \=     Greater than  \>     Question mark \?
%%   Commercial at \@     Left bracket  \[     Backslash     \\
%%   Right bracket \]     Circumflex    \^     Underscore    \_
%%   Grave accent  \`     Left brace    \{     Vertical bar  \|
%%   Right brace   \}     Tilde         \~}
%%
%
% \CheckSum{0}
%
% \iffalse meta-comment
%</!(cfg|fd)>
%
%<*driver>
\ProvidesFile{BHCexam.dtx}
%</driver>
%
%<cls>\NeedsTeXFormat{LaTeX2e}[1995/12/01]
%<cls>\ProvidesClass{BHCexam}
%<cfg>\ProvidesFile{BHCexam.cfg}
  [2011/07/27 v0.2 BHCexam
%<cls>   document class]
%<cfg>   configuration file]
%
%<*driver>
   bundle source file]
%</driver>
%
%<*driver>
\documentclass[a4paper]{ltxdoc}
\usepackage{ctex}
\usepackage{hyperref}
\usepackage{amsmath,amssymb}
 \topmargin 0.5 true cm
 \oddsidemargin 1 true cm
 \evensidemargin 1 true cm
 \textheight 21 true cm
 \textwidth 14 true cm
\EnableCrossrefs
 %\DisableCrossrefs % Say \DisableCrossrefs if index is ready
\CodelineIndex
\RecordChanges      % Gather update information
 %\OnlyDescription  % comment out for implementation details
 %\OldMakeindex     % use if your MakeIndex is pre-v2.9
\hypersetup{colorlinks,linkcolor=blue,citecolor=blue}
\begin{document}  
 \DocInput{BHCexam.dtx}
\end{document}
%</driver>
%
% \fi
%
%
% \changes{v0.0}{2011/07/22}{Initial version}
% \changes{v0.1}{2011/07/23}{version 0.1}
% \changes{v0.2}{2011/07/27}{version 0.2}
%
%
% \DoNotIndex{\begin,\end,\begingroup,\endgroup}
% \DoNotIndex{\ifx,\ifdim,\ifnum,\ifcase,\else,\or,\fi}
% \DoNotIndex{\let,\def,\xdef,\newcommand,\renewcommand}
% \DoNotIndex{\expandafter,\csname,\endcsname,\relax,\protect}
% \DoNotIndex{\Huge,\huge,\LARGE,\Large,\large,\normalsize}
% \DoNotIndex{\small,\footnotesize,\scriptsize,\tiny}
% \DoNotIndex{\normalfont,\bfseries,\slshape,\interlinepenalty}
% \DoNotIndex{\hfil,\par,\vskip,\vspace,\quad}
% \DoNotIndex{\centering,\raggedright}
% \DoNotIndex{\c@secnumdepth,\@startsection,\@setfontsize}
% \DoNotIndex{\ ,\@plus,\@minus,\p@,\z@,\@m,\@M,\@ne,\m@ne}
% \DoNotIndex{\@@par}
%
%
% \GetFileInfo{BHCexam.dtx}
%
%
% \MakeShortVerb{\|}
% \setcounter{StandardModuleDepth}{1}
%
%
% \newcommand{\ctex}{\texttt{ctex}}
% \newcommand{\BHCexam}{\texttt{BHCexam}}
% \newcommand{\exam}{\texttt{exam}}
% \newcommand{\colin}{\texttt{colinexam}}
%
%
% \setlength{\parskip}{0.75ex plus .2ex minus .5ex}
% \renewcommand{\baselinestretch}{1.2}
%
% \newcommand{\rc}{\text{C}}
% \newcommand{\ri}{\text{i}}
% \newcommand{\ra}{\text{A}}
% \newcommand{\rd}{\text{d}}
% \renewcommand\m{\ensuremath{\text{m}}}
% \newcommand\tian{\ensuremath{\text{d}}}
% \newcommand\A{\ensuremath{\text{A}}}
% \newcommand\g{\ensuremath{\text{g}}}
% \newcommand\kg{\ensuremath{\text{kg}}}
% \newcommand\degree{\ensuremath{^\circ}}
% \newcommand\ssd{\ensuremath{\text{\textcelsius}}}
% \newcommand\rad{\ensuremath{\text{rad}}}
% \newcommand\N{\ensuremath{\text{N}}}
% \newcommand\Pa{\ensuremath{\text{Pa}}}
% \newcommand\J{\ensuremath{\text{J}}}
% \newcommand\W{\ensuremath{\text{W}}}
% \newcommand\ohm{\ensuremath{\Omega}}
% \newcommand\mol{\ensuremath{\text{mol}}}
% \newcommand\K{\ensuremath{\text{K}}}
% \newcommand\h{\ensuremath{\text{h}}}
% \newcommand\ton{\ensuremath{\text{t}}}
% \newcommand\squarem{\ensuremath{\text{m$^2$}}}
% \newcommand\cubicm{\ensuremath{\text{m$^3$}}}
% \newcommand\cm{\ensuremath{\text{cm}}}
% \newcommand\mm{\ensuremath{\text{mm}}}
% \newcommand\squarecm{\ensuremath{\text{cm$^2$}}}
% \newcommand\cubiccm{\ensuremath{\text{cm$^3$}}}
% \newcommand\squaremm{\ensuremath{\text{mm$^2$}}}
% \newcommand\cubicmm{\ensuremath{\text{mm$^3$}}}
% \newcommand\liter{\ensuremath{\text{L}}}
% \newcommand{\abs}[1]{\left|#1\right|}
% \newcommand\arccot{\mathop{\text{arccot}}}
%
% \makeatletter
% \def\parg#1{\mbox{$\langle${\it #1\/}$\rangle$}}
% \def\@smarg#1{{\tt\string{}\parg{#1}{\tt\string}}}
% \def\@marg#1{{\tt\string{}{\rm #1}{\tt\string}}}
% \def\marg{\@ifstar\@smarg\@marg}
% \def\@soarg#1{{\tt[}\parg{#1}{\tt]}}
% \def\@oarg#1{{\tt[}{\rm #1}{\tt]}}
% \def\oarg{\@ifstar\@soarg\@oarg}
% \makeatother
%
%
% \title{\bf \BHCexam~宏包说明\thanks
%   {这是鲍宏昌发布的第一个~\LaTeX~宏包。本文件版本号为~\fileversion{},最后修改日期~\filedate{}。}}
% \author{\it 鲍宏昌\thanks{charley792@gmail.com}}
% \date{\small 打印日期:~\today}
% \maketitle
%
% \begin{abstract}
% \BHCexam~宏包提供了一个中学试卷排版的~\LaTeX{}~文档类。
%
% \BHCexam~主要文件包括~\texttt{BHCexam.cls}~文档类和配置文件
% ~\texttt{BHCexam.cfg}。
%
% \BHCexam~宏包由鲍宏昌制作并负责维护。
% \end{abstract}
%
% \tableofcontents
% \newpage
%
% \section{简介}
%
% 本宏包以~\exam~为底层文档类,部分源代码来自于盖鹤麟开发的
% ~\colin。不知道什么原因盖鹤麟自2004年就一直没有发布更新,
% ~\colin~仍然使用CCT实现中文支持,而缺乏对~XeTeX~的支持。2011年7月,
% 鲍宏昌在~\colin~的基础上改用~\ctex~实现中文支持,采用UTF8编码同时
% 支持~XeLaTeX~和~pdfTeX~进行编译,并使用~\texttt{doc}~和
% ~\texttt{docstrip}~工具编写了这个文档,增加了一些新的功能,
% 并把新的宏包命名为~\BHCexam。
%
% 本宏包延续了~\colin~和~\exam~的特点,能让一个刚刚接触
% ~\LaTeX~的初学者,也能轻松用它来排版试卷。希望~\BHCexam~能提高中学
% 教师的工作效率,并把注意力放在试卷的内容上。
% 
% \BHCexam~由两个主要文件构成:文档类~\texttt{BHCexam.cls}~和配置文件
% ~\texttt{BHCexam.cfg}~。后者定义了一些常用的参数。
%
% {\kaishu
% 这两个文件可以通过用~XeLaTeX~编译~\texttt{BHCexam.ins}~文件来得到,
% 而这份说明文档可以通过用~XeLaTeX~编译~\texttt{BHCexam.dtx}~文件来得到。
% 编译说明文档需要~\ctex{}~宏包,为了生成正确的索引和版本记录,
% 需要使用如下命令}
% \begin{verbatim}
% makeindex -s gind.ist -o BHCexam.ind BHCexam.idx
% makeindex -s gglo.ist -o BHCexam.gls BHCexam.glo
% \end{verbatim}
%
% \section{一个简单的例子}
%
% 用~\BHCexam~要排版一张基本的试卷其实很简单。如果你准备对试卷的排版进行
% 更细致的设置,那么请参考~\exam~的文档。
%
% \subsection{\texttt{documentclass} 命令}
% \label{sec:BasicDocumentclass}
%
% 要使用~\BHCexam~文档类,你的\verb"\documentclass" 命令应该是
% \begin{verbatim}
%  \documentclass{BHCexam}
% \end{verbatim}
% 如果,你想使用小四字体作为缺省字体大小,那么添加选项\verb"cs4size"
% \begin{verbatim}
% \documentclass[cs4size]{BHCexam}
% \end{verbatim}
% 更多的选项,请参考\ref{sec:Options}。
%
% \subsection{打印标题和考试须知}
% \DescribeMacro{\maketitle}
% \DescribeMacro{\notice}
% 在试卷上打印标题和考试须知
% \begin{verbatim}
%   \maketitle
%   \notice 
% \end{verbatim}
% 关于标题和考试须知中变量的设置,请参考\ref{sec:Variable}。
%
% \subsection{题目}
% \label{sec:Example}
% \DescribeMacro{\question}
% \DescribeMacro{\choice}
% \DescribeMacro{\part}
% \DescribeMacro{\tiankong}
% \DescribeMacro{\xuanze}
% \DescribeMacro{\jianda}
% 在\verb"questions"环境中用\verb"\question"输入题目。
% 在\verb"choices"环境中用\verb"\choice"输入选项。
% 在\verb"parts"环境中用\verb"\part"输入小问。
% 在\verb"questions"环境中用\verb"\tiankong"、\verb"\xuanze"
% 和\verb"\jianda"分别显示填空题、选择题、简答题的提示语。
% \begin{verbatim}
% \begin{questions}
%   \tiankong
%   \question 这是第1道填空题
%   \question 这是第2道填空题
%   \xuanze
%   \question 问题3是一道选择题
%   \begin{choices}
%     \choice 选项1
%     \choice 选项2
%     \choice 选项3
%     \choice 选项4
%   \begin{choices}
%   \jianda
%   \question 问题4是一道简答题
%   \begin{parts}
%   \part 第1小问
%   \part 第2小问
%   \end{parts}
% \end{questions}
% \end{verbatim}
% 关于填空题、选择题、简答题的提示语中几个变量的设置,请参考\ref{sec:Variable}。
% 关于题目的更多内容,请参考\ref{sec:Environment}。
%
% \section{使用帮助}
%
% \subsection{选项}
% \label{sec:Options}
%
% \changes{v0.2}{2011/07/27}{增加UTF8选项以支持pdflatex}
%
% 宏包的选项用于改变一些缺省的设置。虽然缺省的设置基本能过满足一般用户的
% 使用需要,但用户也可以根据自己的情况,使用这些选项。
%
% \begin{description}
% \item[cs4size]     使用小四字号为缺省字体大小。
% \item[c5size]      使用五号字为缺省字体大小(缺省选项)。
% \item[answers]     在每一个问题后附上答案。
% \item[marginline]  放置装订线。
% \item[16kpaper]    使用16开纸张(缺省使用A4纸张)。
% \item[noindent]    没有缩进。
% \item[printbox]    显示评分框。
% \item[UTF8]        用pdflatex编译时需要。
% \end{description}
%
% \subsection{变量}
% \label{sec:Variable}
% \changes{v0.2}{2011/07/27}{试卷中改用英文标点符号}
% 本宏包在题量和分值等方面均以高考试卷为模板,
% 默认的变量值可以在\texttt{BHCexam.cfg}中设置,当然
% 你也可以在使用相关命令之前使用以下命令进行更改。\\\\
% \DescribeMacro{\biaoti}
% 设置标题信息。
% \begin{quote}
% |\biaoti|\marg*{TEXT}
% \end{quote}
% \DescribeMacro{\kemu}
% 设置科目信息,他会显示在标题下方和页脚内。
% \begin{quote}     
% |\kemu|\marg*{TEXT}
% \end{quote}
% \DescribeMacro{\xinxi}
% 设置总分和考试时间信息,\parg{num1}为总分,\parg{num2}为考试时间。
% \begin{quote}
% |\xinxi|\marg*{num1}\marg*{num2}
% \end{quote}
% \DescribeMacro{\settk}
% 设置填空题的总分、题量和小分信息,\parg{num1}为总分,\parg{num2}为题量,
% \parg{num3}为小分。
% \begin{quote}
% |\settk|\marg*{num1}\marg*{num2}\marg*{num3}
% \end{quote}
% \DescribeMacro{\setxz}
% 设置选择题的总分、题量和小分信息,\parg{num1}为总分,\parg{num2}为题量,
% \parg{num3}为小分。
% \begin{quote}
% |\setxz|\marg*{num1}\marg*{num2}\marg*{num3}
% \end{quote}
% \DescribeMacro{\setjd}
% 设置简答题的总分、题量和小分信息,\parg{num1}为总分,\parg{num2}为题量。
% \begin{quote}
% |\setjd|\marg*{num1}\marg*{num2}\marg*{num3}
% \end{quote}

% \subsection{环境}
% \label{sec:Environment} 
% 经常使用的环境有\verb"questions"环境、\verb"choices"环境、\verb"parts"环境,
% 关于它们的简单介绍,请参考\ref{sec:Example},这里做一点补充说明,
% 更详细的介绍,请参考~\exam~文档。\\\\
% 在排版简答题时需要用\verb"\part"命令输入各小问的分值,宏包会自动算出总分并显示在该简答题的第一行。
% 当该道简答题没有小问时,则要用\verb"\question"命令输入该问题的分值。
% 
% \begin{verbatim}
%  ...
%  \jianda
%  \question 这是一道简答题
%  \begin{parts}[
%  \part[3] 第1小问3分。
%  \part[3] 第2小问3分。
%  \part[3] 第3小问4分。
%  \end{parts}
%  \question[12] 这是一道没有小问的简答题,这道题有12分
%  ... 
% \end{verbatim}
% 在\verb"\question"后新建\verb"solution"环境,在其中输入该问题的解答,
% 在不显示答案的情况下,该问题后会预留答题空间。
% \begin{verbatim}
%  ...
%  \jianda
%  \question 这是一道简答题
%  \begin{solution}
%  这是解答,不显示答案的情况下则这个问题后预留答题空间。
%  \end{solution}
%  ... 
% \end{verbatim}
%
% \subsection{常用命令}
% 
% \DescribeMacro{\newpage}
% 每道问题的间距是弹性设置的,你只要在想换页的地方输入\verb"\newpage"命令,
% 则上一页的各问题间距会自动调整到最美观的效果。\\
% \DescribeMacro{\mininotice}
% 在一行内输出精简的考试注意事项。\\
% \DescribeMacro{\printmalol}
% 在当前页为正面时,在左边插入装订线(仅在使用marginline选项时有效)。\\
% \DescribeMacro{\printmalol}
% 在当前页为反面时,在左边插入装订线(仅在使用marginline选项时有效)。\\

% \subsection{数学符号}
% \begin{tabular}{ll}
% \hline
% \verb"\oneb" & 生成一条很小的供排版填空题空格用的横线\\\hline
% \verb"\twob" & 生成一条长一些的供排版填空题空格用的横线\\\hline
% \verb"\fourb" & 生成一长更长的供排版填空题空格用的横线\\\hline
% \verb"\sixb" & 生成一长较长的供排版填空题空格用的横线\\\hline
% \verb"\tenb" & 生成一条很长的供排版填空题空格用的横线\\\hline
% \verb"\rc" & 在数学模式下输入正体的~C(组合数符号)\\\hline
% \verb"\ra" & 在数学模式下输入正体的~A(排列数符号)\\\hline
% \verb"\ri" & 在数学模式下输入正体的~i(虚数单位)\\\hline
% \verb"\abs{...}" & 用以输入绝对值\\\hline
% \verb"\arccot" & 在数学模式下输入$\arccot$\\\hline
% \verb"\m" & 在文本模式和数学模式下均可使用,生成单位符号~\m\\\hline
% \verb"\W" & 在文本模式攻数学模式下均可使用,生成单位符号~\W\\\hline
% \verb"\A" & 在文本模式和数学模式下均可使用,生成单位符号~\A\\ \hline
% \verb"\g" & 在文本模式和数学模式下均可使用,生成单位符号~\g\\ \hline
% \verb"\kg" & 在文本模式和数学模式下均可使用,生成单位符号~\kg\\ \hline
% \verb"\degree" & 在文本模式和数学模式下均可使用,生成单位符号~\degree\\\hline
% \verb"\ssd" & 在文本模式和数学模式下均可使用,生成单位符号~\ssd\\\hline
% \verb"\rad" & 在文本模式和数学模式下均可使用,生成单位符号~\rad\\\hline
% \verb"\N" & 在文本模式和数学模式下均可使用,生成单位符号~\N\\\hline
% \verb"\Pa" & 在文本模式和数学模式下均可使用,生成单位符号~\Pa\\\hline
% \verb"\J" & 在文本模式和数学模式下均可使用,生成单位符号~\J\\\hline
% \verb"\ohm" & 在文本模式和数学模式下均可使用,生成单位符号~\ohm\\\hline
% \verb"\mol" & 在文本模式和数学模式下均可使用,生成单位符号~\mol\\\hline
% \verb"\K" & 在文本模式和数学模式下均可使用,生成单位符号~\K\\\hline
% \verb"\h" & 在文本模式和数学模式下均可使用,生成单位符号~\h\\\hline
% \verb"\ton" & 在文本模式和数学模式下均可使用,生成单位符号~\ton\\\hline
% \verb"\squarem" & 在文本模式和数学模式下均可使用,生成单位符号~\squarem\\\hline
% \verb"\cubicm" & 在文本模式和数学模式下均可使用,生成单位符号~\cubicm\\\hline
% \verb"\cm" & 在文本模式和数学模式下均可使用,生成单位符号~\cm\\\hline
% \verb"\mm" & 在文本模式和数学模式下均可使用,生成单位符号~\mm\\\hline
% \verb"\squarecm" & 在文本模式和数学模式下均可使用,生成单位符号~\squarecm\\\hline
% \verb"\cubiccm" & 在文本模式和数学模式下均可使用,生成单位符号~\cubiccm\\\hline
% \verb"\squaremm" & 在文本模式和数学模式下均可使用,生成单位符号~\squaremm\\\hline
% \verb"\cubicmm" & 在文本模式和数学模式下均可使用,生成单位符号~\cubicmm\\\hline
% \verb"\liter" & 在文本模式和数学模式下均可使用,生成单位符号~\liter\\\hline
% \end{tabular}
%
% \StopEventually{
% } ^^A end StopEventually
%
% \section{源代码说明}
%
% \subsection{选项}
%
%\begin{macro}{\input}
% \changes{v0.2}{2011/07/27}{改用input语句导入ctex类文件}
% 导入ctex类的选项
%    \begin{macrocode}
%<*cls>
\input{ctex-common-opts.def}
\input{ctex-class-opts.def}
\input{ctex-caption-opts.def}
%</cls>
%    \end{macrocode}
%\end{macro}
%
% 处理~\BHCexam~文档类的选项
%
% \begin{macro}{\@sixteenkpaper}
% 16k纸张大小设置,缺省选项为a4paper
%    \begin{macrocode}
%<*cls>
\newif\if@sixteenkpaper \@sixteenkpapertrue
\DeclareOption{16kpaper}{\@sixteenkpaperfalse}
%</cls>
%    \end{macrocode}
% \end{macro}
%
% \begin{macro}{\@marginline}
% 是否有装订线
%    \begin{macrocode}
%<*cls>
\newif\if@marginline \@marginlinefalse
\DeclareOption{marginline}{\@marginlinetrue}
%</cls>
%    \end{macrocode}
% \end{macro}
%
% 不缩进,缺省为缩进
%    \begin{macrocode}
%<*cls>
\newif\if@noindent \@noindentfalse
\DeclareOption{noindent}{\@noindenttrue}
%</cls>
%    \end{macrocode}
%
% 显示答案的方式,缺省不显示答案
%    \begin{macrocode}
%<cls>\DeclareOption{answers}{\PassOptionsToClass{\CurrentOption}{exam}}
%    \end{macrocode}

% \begin{macro}{\@printbox}
% 显示计分框,缺省为不显示。
%    \begin{macrocode}
%<*cls>
\newif\if@printbox \@printboxfalse
\DeclareOption{printbox}{\@printboxtrue}
%</cls>
%    \end{macrocode}
% \end{macro}
%
%
% 把没有定义的选项传递给缺省的文档类
%    \begin{macrocode}
%<cls>\DeclareOption*{\PassOptionsToClass{\CurrentOption}{exam}}
%    \end{macrocode}
%
% 处理选项
%    \begin{macrocode}
%<cls>\ProcessOptions
%    \end{macrocode}
%
% 装入缺省的文档类
%    \begin{macrocode}
%<cls>\LoadClass[addpoints]{exam}
%    \end{macrocode}
% 
% 导入ctex类的实现
%    \begin{macrocode}
%<*cls>
\input{ctex-common.def}
\input{ctex-caption.def}
\input{ctex-class.def}
%</cls>
%    \end{macrocode}
% \subsection{宏包}
%
% \begin{macro}{\RequirePackage}
% 我们需要使用的一些宏包
%    \begin{macrocode}
%<*cls>
\RequirePackage{amsmath,amssymb,bm}
\RequirePackage{graphicx}
\RequirePackage{ifpdf,ifxetex}
%</cls>
%    \end{macrocode}
% 
% 用geometry宏包进行页面设置
% \changes{v0.2}{2011/07/27}{改用geometry宏包实现纸张设置}
%    \begin{macrocode}
%<*cls>
\if@marginline
\if@sixteenkpaper
\RequirePackage[papersize={184mm,260mm},hmargin={3cm,2cm},
vmargin={2cm,2cm},marginparsep=0.5cm,hoffset=0cm,voffset=0cm,
footnotesep=0.5cm,headsep=0.5cm,twoside]{geometry}
\else
\RequirePackage[paper=a4paper,hmargin={3cm,2cm},vmargin={2cm,2cm},
marginparsep=0.5cm,hoffset=0cm,voffset=0cm,footnotesep=0.5cm,
headsep=0.5cm,twoside]{geometry}
\fi
\else
\if@sixteenkpaper
\RequirePackage[papersize={184mm,260mm},hmargin={2cm,2cm},
vmargin={2cm,2cm},marginparsep=0.5cm,hoffset=0cm,voffset=0cm,
footnotesep=0.5cm,headsep=0.5cm]{geometry}
\else
\RequirePackage[paper=a4paper,hmargin={2cm,2cm},vmargin={2cm,2cm},
marginparsep=0.5cm,hoffset=0cm,voffset=0cm,footnotesep=0.5cm,
headsep=0.5cm]{geometry}
\fi
\fi
%</cls>
%    \end{macrocode}
%\end{macro}
%
% \subsection{自定义设置}
%行距、页眉、页脚
%    \begin{macrocode}
%<*cls>
\renewcommand{\baselinestretch}{1.5}
\pagestyle{headandfoot}
%\runningheadrule
%\firstpageheadrule
%\runningfootrule
%\firstpagefootrule
\header{}{}{}
\footer{}{\small \kaishu{\@kemu}\quad 第~\thepage~页(共~\numpages~页)}{}
%</cls>
%    \end{macrocode}
%
% 分值显示
%    \begin{macrocode}
%<*cls>
\pointname{分}
\pointformat{\kaishu (\thepoints)}
%</cls>
%    \end{macrocode}
%
% 解的显示
%    \begin{macrocode}
%<*cls>
\renewcommand{\solutiontitle}{\noindent\heiti{解:}\noindent}
%</cls>
%    \end{macrocode}
%
% 小问的显示
%    \begin{macrocode}
%<*cls>
\renewcommand{\thepartno}{\arabic{partno}}
\renewcommand{\partlabel}{(\thepartno)}
\renewcommand{\partshook}{
  \settowidth{\leftmargin}{(3).\hskip\labelsep}
  \if@noindent \setlength\leftmargin{0pt} \fi
}
%</cls>
%    \end{macrocode}
%
% 选项的显示
%    \begin{macrocode}
%<*cls>
\renewcommand{\choiceshook}{
  \settowidth{\leftmargin}{W.\hskip\labelsep}
  \if@noindent \setlength\leftmargin{0pt} \fi
}
%</cls>
%    \end{macrocode}
%
% 解答的显示
%    \begin{macrocode}
%<*cls>
\renewenvironment{solution}%
  {%
    \ifprintanswers
      \begingroup
      \Solution@Emphasis
      \begin{TheSolution}%
    \else
      \ifcancelspace
        % Do nothing
      \else
        \par
        \penalty 0
        \vfill%
        \if@printbox \if@houpinfen \houpinfen \fi \fi
      \fi
      \setbox\z@\vbox\bgroup
    \fi
  }{%
    \ifprintanswers
      \end{TheSolution}%
      \endgroup
    \else
      \egroup
    \fi
  }%
%</cls>
%    \end{macrocode}
%
% \subsection{新的命令和环境}
%
% \begin{macro}{\printmlor}
% \begin{macro}{\printmlol}
% \changes{v0.1}{2011/07/23}{支持在首页插入装订线}
% \changes{v0.2}{2011/07/27}{手动在指定页插入左(右)装订线}
% 装订线
%    \begin{macrocode}
%<*cls>
\if@marginline 
\newsavebox{\zdxl}
\sbox{\zdxl}{
\begin{minipage}{0.7\paperheight} 
\begin{center}
\heiti 班级\underline{\hspace{15ex}} \quad
姓名 \underline{\hspace{15ex}} \quad
学号 \underline{\hspace{15ex}} \quad \\ 
\vspace{3ex}
\dotfill 装 \dotfill 订 \dotfill 线 \dotfill 
\end{center}
\end{minipage}
}
\newsavebox{\zdxr}
\sbox{\zdxr}{
\begin{minipage}{0.7\paperheight} 
\begin{center}
\heiti \hfill 请 \hfill 不 \hfill 要 \hfill 在 \hfill
 装 \hfill 订 \hfill 线 \hfill 内 \hfill 答 \hfill 题 \hfill \\ 
\vspace{3ex}
\dotfill 装 \dotfill 订 \dotfill 线 \dotfill 
\end{center}
\end{minipage}
}
\newcommand{\printmlol}{
\marginpar{\rotatebox{90}{\usebox{\zdxl}}}
}
\newcommand{\printmlor}{
\marginpar{\rotatebox{-90}{\usebox{\zdxr}}}
}
\reversemarginpar
\fi
%</cls>
%    \end{macrocode}
% \end{macro}
% \end{macro}
%
% \begin{macro}{\biaoti}
% \begin{macro}{\kemu}
% 标题
%    \begin{macrocode}
%<*cfg>
\def\@biaoti{2011年全国普通高等学校招生统一考试(上海卷)}
\def\@kemu{数学(理科)}
%</cfg>
%<*cls>
\newcommand\biaoti[1]{\def\@biaoti{#1}}
\newcommand\kemu[1]{\def\@kemu{#1}}
\renewcommand\maketitle{
  \begin{center}{\heiti \Large{\@biaoti}}\end{center}
  \begin{center}{\heiti \Large{\@kemu}}\end{center}
}
%</cls>
%    \end{macrocode}
% \end{macro}
% \end{macro}
%
% \begin{macro}{\mininotice}
% \begin{macro}{\xinxi}
% 一行内显示考试时间和考试总分
%    \begin{macrocode}{\mininotice}
%<*cfg>
\def\@zongfen{150}
\def\@shijian{120}
%</cfg>
%<*cls>
\newcommand\xinxi[2]{
  \def\@zongfen{#1}
  \def\@shijian{#2}
}
\newcommand\mininotice{
  \begin{center}{
    \kaishu (本试卷满分~\@zongfen~分, 考试时间~\@shijian~分钟)}
  \end{center}
}
%</cls>
%    \end{macrocode}
% \end{macro}
% \end{macro}
%
% \begin{macro}{\notice}
% 注意事项
%    \begin{macrocode}
%<*cls>
\newcommand{\notice}{
  \heiti 注意事项: \songti
  \begin{enumerate}
  \item 答卷前, 考生务必将姓名、高考准考证号、校验码等填写清楚.
  \item 本试卷共~\numquestions{}~道试题, 满分~\@zongfen~分,考试时间~\@shijian~分钟.
  \end{enumerate}
}
%</cls>
%    \end{macrocode}
%\end{macro}
%
% \begin{macro}{\pingfen}
% 前评分框
%    \begin{macrocode}
%<*cls>
\newlength\@boxwidth
\setlength\@boxwidth{0ex}
\if@printbox \setlength\@boxwidth{18ex} \fi
\newcommand\pinfen{
  \heiti
  \begin{minipage}{\@boxwidth}
  \begin{tabular}{|c|c|}
  \hline
  得分 & 评卷人\\
  \hline
      &       \\ 
  \hline
  \end{tabular}
  \end{minipage}
}
%</cls>
%    \end{macrocode}
% \end{macro}
%
% \begin{macro}{\houpinfen}
% 后评分框
%    \begin{macrocode}
%<*cls>
\newcommand{\houpinfen}{
  \hfill
  \begin{tabular}{|l|l|}
    \hline
    得分 & \hspace*{1.5cm}\\
    \hline
  \end{tabular}
  \bigskip
}
%</cls>
%    \end{macrocode}
% \end{macro}
%
% \begin{macro}{\oneb}
% \begin{macro}{\twob}
% \begin{macro}{\sixb}
% \begin{macro}{\tenb}
% 空格
%    \begin{macrocode}
%<*cls>
\newcommand\oneb{\underline{\hspace{1em}}\hspace{0.001em}}
\newcommand\twob{\oneb\oneb}
\newcommand{\sixb}{\twob\twob}
\newcommand\tenb{\twob\twob\twob\twob\twob}
%</cls>
%    \end{macrocode}
% \end{macro}
% \end{macro}
% \end{macro}
% \end{macro}
%
% 填空题、选择题、简答题
%    \begin{macrocode}
%<*cfg>
\def\@tiankong@zongfen{56}
\def\@tiankong@tishu{14}
\def\@tiankong@fen{4}
\def\@xuanze@zongfen{16}
\def\@xuanze@tishu{4}
\def\@xuanze@fen{4}
\def\@jianda@zongfen{78}
\def\@jianda@tishu{5}
%</cfg>
%<*cls>
\newcounter{@dati}
\newif\if@houpinfen \@houpinfenfalse
\newcommand\settk[3]{
  \def\@tiankong@zongfen{#1}
  \def\@tiankong@tishu{#2}
  \def\@tiankong@fen{#3}
}
\newcommand\tiankong{
  \@houpinfenfalse
  \stepcounter{@dati} 
  \fullwidth{
    \if@printbox \pinfen \fi
    \begin{minipage}{\textwidth-\@boxwidth}
    \heiti \chinese{@dati}. 填空题(\kaishu 本大题满分~\@tiankong@zongfen~分) \heiti 本大题有~\@tiankong@tishu~题, 考生应在答题纸相应编号的空格内直接写结果, 每个空格填对得~\@tiankong@fen~分, 否则一律得零分.
    \end{minipage}
  }
}
\newcommand\setxz[3]{
  \def\@xuanze@zongfen{#1}
  \def\@xuanze@tishu{#2}
  \def\@xuanze@fen{#3}
}
\newcommand\xuanze{
  \@houpinfenfalse
  \stepcounter{@dati} 
  \fullwidth{
    \if@printbox \pinfen \fi
    \begin{minipage}{\textwidth-\@boxwidth}  
      \heiti \chinese{@dati}. 选择题(\kaishu 本大题满分~\@xuanze@zongfen~分) \heiti 本大题共有~\@xuanze@tishu~题, 每题有且只有一个正确答案, 考生应在答题纸的相应编号上, 将代表答案的小方格涂黑, 选对得~\@xuanze@fen~分, 否则一律得零分.
    \end{minipage}
  }
}
\newcommand\setjd[2]{
  \def\@jianda@zongfen{#1}
  \def\@jianda@tishu{#2}
}
\newcommand\jianda{
  \@houpinfentrue
  \qformat{\hskip\labelsep \kaishu \thequestion.~~(本题满分~\totalpoints~分)\hfill}
  \stepcounter{@dati}
  \fullwidth{
    \if@printbox \pinfen \fi  
    \begin{minipage}{\textwidth-\@boxwidth}
      \heiti \chinese{@dati}. 简答题(\kaishu 本大题满分~\@jianda@zongfen~分)~\heiti 本大题共有~\@jianda@tishu~题, 解答下列各题必须在答题纸相应的编号规定区域内写出必要的步骤.
    \end{minipage}
  }
}
%</cls>
%    \end{macrocode}
%
% 数学运算符号、单位
%    \begin{macrocode}
%<*cls>
\newcommand{\rc}{\text{C}}
\newcommand{\ri}{\text{i}}
\newcommand{\ra}{\text{A}}
\newcommand{\rd}{\text{d}}
\newcommand\tian{\ensuremath{\text{d}}}
\newcommand\A{\ensuremath{\text{A}}}
\def\m{\ensuremath{\text{m}}}
\newcommand\g{\ensuremath{\text{g}}}
\newcommand\kg{\ensuremath{\text{kg}}}
\newcommand\degree{\ensuremath{^\circ}}
\newcommand\ssd{\ensuremath{\text{\textcelsius}}}
\newcommand\rad{\ensuremath{\text{rad}}}
\newcommand\N{\ensuremath{\text{N}}}
\newcommand\Pa{\ensuremath{\text{Pa}}}
\newcommand\J{\ensuremath{\text{J}}}
\newcommand\W{\ensuremath{\text{W}}}
\newcommand\ohm{\ensuremath{\Omega}}
\newcommand\mol{\ensuremath{\text{mol}}}
\newcommand\K{\ensuremath{\text{K}}}
\newcommand\h{\ensuremath{\text{h}}}
\newcommand\ton{\ensuremath{\text{t}}}
\newcommand\squarem{\ensuremath{\text{m$^2$}}}
\newcommand\cubicm{\ensuremath{\text{m$^3$}}}
\newcommand\cm{\ensuremath{\text{cm}}}
\newcommand\mm{\ensuremath{\text{mm}}}
\newcommand\squarecm{\ensuremath{\text{cm$^2$}}}
\newcommand\cubiccm{\ensuremath{\text{cm$^3$}}}
\newcommand\squaremm{\ensuremath{\text{mm$^2$}}}
\newcommand\cubicmm{\ensuremath{\text{mm$^3$}}}
\newcommand\liter{\ensuremath{\text{L}}}
\newcommand{\abs}[1]{\left|#1\right|}
\newcommand\arccot{\mathop{\text{arccot}}}
\newcommand\pingxing{\parallel}
%<\cls>
%    \end{macrocode}
%
% 读取配置文件
%    \begin{macrocode}
%<*cls>
\AtEndOfPackage{\makeatletter\input{BHCexam.cfg}\makeatother}
%</cls>
%    \end{macrocode}
%
%

% \Finale
%
% \setcounter{IndexColumns}{2}
% \IndexPrologue{\section*{索引}}
%
% \GlossaryPrologue{\section*{版本更新}}
%
% \PrintIndex \PrintChanges
\endinput

\EnableCrossrefs
\makeatother}
%</cls>
%    \end{macrocode}
%
%

% \Finale
%
% \setcounter{IndexColumns}{2}
% \IndexPrologue{\section*{索引}}
%
% \GlossaryPrologue{\section*{版本更新}}
%
% \PrintIndex \PrintChanges
\endinput

\EnableCrossrefs
\makeatother}
%</cls>
%    \end{macrocode}
%
%

% \Finale
%
% \setcounter{IndexColumns}{2}
% \IndexPrologue{\section*{索引}}
%
% \GlossaryPrologue{\section*{版本更新}}
%
% \PrintIndex \PrintChanges
\endinput

\EnableCrossrefs
\makeatother}
%</cls>
%    \end{macrocode}
%
%

% \Finale
%
% \setcounter{IndexColumns}{2}
% \IndexPrologue{\section*{索引}}
%
% \GlossaryPrologue{\section*{版本更新}}
%
% \PrintIndex \PrintChanges
\endinput

\EnableCrossrefs
